%%%%Packages
\pdfoutput=1
\documentclass[11pt, a4paper, twoside,leqno]{amsart}
\usepackage[centering, totalwidth = 380pt, totalheight = 600pt]{geometry}
\usepackage{nag}
\usepackage{amssymb, amsmath, amsthm}
\usepackage{microtype, stmaryrd, url, lmodern, eucal, bbold}
\usepackage[shortlabels]{enumitem}
%For \yo: 
\usepackage[latin1,utf8x]{inputenc}
\usepackage{mathrsfs}
\usepackage{ucs}
%\usepackage[utf8x]{inputenc}
\DeclareFontFamily{U}{min}{}
\DeclareFontShape{U}{min}{m}{n}{<-> udmj30}{}
%%%Plus macro further down:
%\newcommand{\yo}{\text{\usefont{U}{min}{m}{n}\symbol{'210}}} 
\usepackage{color}
\definecolor{darkgreen}{rgb}{0,0.45,0}
\usepackage[colorlinks,citecolor=darkgreen,linkcolor=darkgreen]{hyperref}
\usepackage[british]{babel}

%%%%Spacing

\lineskiplimit=-3pt
\makeatletter
%\def\@cite#1#2{[{#1\if@tempswa , #2\fi}]}% OLD
\def\@cite#1#2{[{#1\if@tempswa ,~#2\fi}]}% NEW
\makeatother

%%%%Colours

\def\blue{\leavevmode\color{blue}}%
\def\black{\leavevmode\color{black}}%
\def\red{\leavevmode\color{red}}%
\def\green{\leavevmode\color{green}}%

%%%%Fonts

\renewcommand{\bfdefault}{b}
\DeclareMathAlphabet{\mathbf}{OT1}{cmr}{b}{n}
\DeclareFontSeriesDefault[rm]{bf}{b}

%%%%Xypic

\usepackage[arrow, matrix, tips, curve, graph, rotate]{xy}
\SelectTips{cm}{10}

\makeatletter
\def\matrixobject@{%
  \edef \next@{={\DirectionfromtheDirection@ }}%
  \expandafter \toks@ \next@ \plainxy@
  \let\xy@@ix@=\xyq@@toksix@
  \xyFN@ \OBJECT@}
\let\xy@entry@@norm=\entry@@norm
\def\entry@@norm@patched{%
  \let\object@=\matrixobject@
  \xy@entry@@norm }
\AtBeginDocument{\let\entry@@norm\entry@@norm@patched}
\makeatother

\newcommand{\twocong}[2][0.5]{\ar@{}[#2] \save ?(#1)*{\cong}\restore}
\newcommand{\twoeq}[2][0.5]{\ar@{}[#2] \save ?(#1)*{=}\restore}
\newcommand{\rtwocell}[3][0.5]{\ar@{}[#2] \ar@{=>}?(#1)+/l 0.2cm/;?(#1)+/r 0.2cm/^{#3}}
\newcommand{\ltwocell}[3][0.5]{\ar@{}[#2] \ar@{=>}?(#1)+/r 0.2cm/;?(#1)+/l 0.2cm/^{#3}}
\newcommand{\ltwocello}[3][0.5]{\ar@{}[#2] \ar@{=>}?(#1)+/r 0.2cm/;?(#1)+/l 0.2cm/_{#3}}
\newcommand{\dtwocell}[3][0.5]{\ar@{}[#2] \ar@{=>}?(#1)+/u  0.2cm/;?(#1)+/d 0.2cm/^{#3}}
\newcommand{\dltwocell}[3][0.5]{\ar@{}[#2] \ar@{=>}?(#1)+/ur  0.2cm/;?(#1)+/dl 0.2cm/^{#3}}
\newcommand{\drtwocell}[3][0.5]{\ar@{}[#2] \ar@{=>}?(#1)+/ul  0.2cm/;?(#1)+/dr 0.2cm/^{#3}}
\newcommand{\dthreecell}[3][0.5]{\ar@{}[#2] \ar@3{->}?(#1)+/u  0.2cm/;?(#1)+/d 0.2cm/^{#3}}
\newcommand{\utwocell}[3][0.5]{\ar@{}[#2] \ar@{=>}?(#1)+/d 0.2cm/;?(#1)+/u 0.2cm/_{#3}}
\newcommand{\dtwocelltarg}[3][0.5]{\ar@{}#2 \ar@{=>}?(#1)+/u  0.2cm/;?(#1)+/d 0.2cm/^{#3}}
\newcommand{\utwocelltarg}[3][0.5]{\ar@{}#2 \ar@{=>}?(#1)+/d  0.2cm/;?(#1)+/u 0.2cm/_{#3}}

\newcommand{\pushoutcorner}[1][dr]{\save*!/#1+1.2pc/#1:(1,-1)@^{|-}\restore}
\newcommand{\pullbackcorner}[1][dr]{\save*!/#1-1.2pc/#1:(-1,1)@^{|-}\restore}

\newdir{(}{{}*!<0em,-.14em>-\cir<.14em>{l^r}}
\newdir{ (}{{}*!/-5pt/\dir{(}}
\newdir{ >}{{}*!/-5pt/\dir{>}}
\newcommand{\sh}[2]{**{!/#1 -#2/}}

%%%%Operators

\DeclareMathOperator{\el}{el}
\DeclareMathOperator{\ob}{ob}
\DeclareMathOperator{\colim}{colim}
\DeclareMathOperator{\mor}{mor}
\DeclareMathOperator{\im}{im}

\newcommand{\cat}[1]{\mathbf{#1}}
\newcommand{\thg}{{\mathord{\text{--}}}}

\newcommand{\abs}[1]{{\left|{#1}\right|}}
\newcommand{\dbr}[1]{\left\llbracket{#1}\right\rrbracket}
\newcommand{\res}[2]{\left.{#1}\right|_{#2}}
\newcommand{\set}[2]{\left{\,#1 \ \vrule\  #2\,\right}}
\newcommand{\spn}[1]{{\langle{#1}\rangle}}
\newcommand{\elt}[1]{\left\llcorner{#1}\right\lrcorner}
\newcommand{\defeq}{\mathrel{\mathop:}=}
\newcommand{\cd}[2][]{\vcenter{\hbox{\xymatrix#1{#2}}}}
\newcommand{\quot}{\delimiter"502F30E\mathopen{}}

%%%%Letters

\renewcommand{\phi}{\varphi}
\newcommand{\A}{{\mathcal A}}
\newcommand{\B}{{\mathcal B}}
\newcommand{\C}{{\mathcal C}}
\newcommand{\D}{{\mathcal D}}
\newcommand{\E}{{\mathcal E}}
\newcommand{\F}{{\mathcal F}}
\newcommand{\G}{{\mathcal G}}
\renewcommand{\H}{{\mathcal H}}
\newcommand{\I}{{\mathcal I}}
\newcommand{\J}{{\mathcal J}}
\newcommand{\K}{{\mathcal K}}
\renewcommand{\L}{{\mathcal L}}
\newcommand{\M}{{\mathcal M}}
\newcommand{\N}{{\mathcal N}}
\renewcommand{\O}{{\mathcal O}}
\renewcommand{\P}{{\mathcal P}}
\newcommand{\Q}{{\mathcal Q}}
\newcommand{\R}{{\mathcal R}}
\let\sec=\S
\renewcommand{\S}{{\mathcal S}}
\newcommand{\T}{{\mathcal T}}
\newcommand{\U}{{\mathcal U}}
\newcommand{\V}{{\mathcal V}}
\newcommand{\W}{{\mathcal W}}
\newcommand{\X}{{\mathcal X}}
\newcommand{\Y}{{\mathcal Y}}
\newcommand{\Z}{{\mathcal Z}}

%%%%Arrows

\newcommand{\xtor}[1]{\cdl[@1]{{} \ar[r]|-{\object@{|}}^{#1} & {}}}
\newcommand{\tor}{\ensuremath{\relbar\joinrel\mapstochar\joinrel\rightarrow}}
\newcommand{\To}{\ensuremath{\Rightarrow}}
\newcommand{\Tor}{\ensuremath{\Relbar\joinrel\Mapstochar\joinrel\Rightarrow}}

\makeatletter

\def\hookleftarrowfill@{\arrowfill@\leftarrow\relbar{\relbar\joinrel\rhook}}
\def\twoheadleftarrowfill@{\arrowfill@\twoheadleftarrow\relbar\relbar}
\def\leftbararrowfill@{\arrowdoublefill@{\leftarrow\mkern-5mu}\relbar\mapstochar\relbar\relbar}
\def\Leftbararrowfill@{\arrowdoublefill@{\Leftarrow\mkern-2mu}\Relbar\Mapstochar\Relbar\Relbar}
\def\leftringarrowfill@{\arrowdoublefill@{\leftarrow\mkern-3mu}\relbar{\mkern-3mu\circ\mkern-2mu}\relbar\relbar}
\def\lefttriarrowfill@{\arrowfill@{\mathrel\triangleleft\mkern0.5mu\joinrel\relbar}\relbar\relbar}
\def\Lefttriarrowfill@{\arrowfill@{\mathrel\triangleleft\mkern1mu\joinrel\Relbar}\Relbar\Relbar}

\def\hookrightarrowfill@{\arrowfill@{\lhook\joinrel\relbar}\relbar\rightarrow}
\def\twoheadrightarrowfill@{\arrowfill@\relbar\relbar\twoheadrightarrow}
\def\rightbararrowfill@{\arrowdoublefill@{\relbar\mkern-0.5mu}\relbar\mapstochar\relbar\rightarrow}
\def\Rightbararrowfill@{\arrowdoublefill@{\Relbar\mkern-2mu}\Relbar\Mapstochar\Relbar\Rightarrow}
\def\rightringarrowfill@{\arrowdoublefill@\relbar\relbar{\mkern-2mu\circ\mkern-3mu}\relbar{\mkern-3mu\rightarrow}}
\def\righttriarrowfill@{\arrowfill@\relbar\relbar{\relbar\joinrel\mkern0.5mu\mathrel\triangleright}}
\def\Righttriarrowfill@{\arrowfill@\Relbar\Relbar{\Relbar\joinrel\mkern1mu\mathrel\triangleright}}

\def\leftrightarrowfill@{\arrowfill@\leftarrow\relbar\rightarrow}
\def\mapstofill@{\arrowfill@{\mapstochar\relbar}\relbar\rightarrow}

\renewcommand*\xleftarrow[2][]{\ext@arrow 20{20}0\leftarrowfill@{#1}{#2}}
\providecommand*\xLeftarrow[2][]{\ext@arrow 60{22}0{\Leftarrowfill@}{#1}{#2}}
\providecommand*\xhookleftarrow[2][]{\ext@arrow 10{20}0\hookleftarrowfill@{#1}{#2}}
\providecommand*\xtwoheadleftarrow[2][]{\ext@arrow 60{20}0\twoheadleftarrowfill@{#1}{#2}}
\providecommand*\xleftbararrow[2][]{\ext@arrow 10{22}0\leftbararrowfill@{#1}{#2}}
\providecommand*\xLeftbararrow[2][]{\ext@arrow 50{24}0\Leftbararrowfill@{#1}{#2}}
\providecommand*\xleftringarrow[2][]{\ext@arrow 10{26}0\leftringarrowfill@{#1}{#2}}
\providecommand*\xlefttriarrow[2][]{\ext@arrow 80{24}0\lefttriarrowfill@{#1}{#2}}
\providecommand*\xLefttriarrow[2][]{\ext@arrow 80{24}0\Lefttriarrowfill@{#1}{#2}}

\renewcommand*\xrightarrow[2][]{\ext@arrow 01{20}0\rightarrowfill@{#1}{#2}}
\providecommand*\xRightarrow[2][]{\ext@arrow 04{22}0{\Rightarrowfill@}{#1}{#2}}
\providecommand*\xhookrightarrow[2][]{\ext@arrow 00{20}0\hookrightarrowfill@{#1}{#2}}
\providecommand*\xtwoheadrightarrow[2][]{\ext@arrow 03{20}0\twoheadrightarrowfill@{#1}{#2}}
\providecommand*\xrightbararrow[2][]{\ext@arrow 01{22}0\rightbararrowfill@{#1}{#2}}
\providecommand*\xRightbararrow[2][]{\ext@arrow 04{24}0\Rightbararrowfill@{#1}{#2}}
\providecommand*\xrightringarrow[2][]{\ext@arrow 01{26}0\rightringarrowfill@{#1}{#2}}
\providecommand*\xrighttriarrow[2][]{\ext@arrow 07{24}0\righttriarrowfill@{#1}{#2}}
\providecommand*\xRighttriarrow[2][]{\ext@arrow 07{24}0\Righttriarrowfill@{#1}{#2}}

\providecommand*\xmapsto[2][]{\ext@arrow 01{20}0\mapstofill@{#1}{#2}}
\providecommand*\xleftrightarrow[2][]{\ext@arrow 10{22}0\leftrightarrowfill@{#1}{#2}}
\providecommand*\xLeftrightarrow[2][]{\ext@arrow 10{27}0{\Leftrightarrowfill@}{#1}{#2}}

\makeatother

%%%%Theorems, etc

\numberwithin{equation}{section}

\theoremstyle{plain}
\newtheorem{Thm}{Theorem}
\newtheorem{Prop}[Thm]{Proposition}
\newtheorem{Cor}[Thm]{Corollary}
\newtheorem{Lemma}[Thm]{Lemma}

\theoremstyle{definition}
\newtheorem{Defn}[Thm]{Definition}
\newtheorem{Not}[Thm]{Notation}
\newtheorem{Ex}[Thm]{Example}
\newtheorem{Exs}[Thm]{Examples}
\newtheorem{Rk}[Thm]{Remark}

\newcommand{\Cat}{\cat{Cat}}
\newcommand{\Set}{\cat{Set}}
\newcommand{\CAT}{\cat{CAT}}
\newcommand{\SET}{\cat{SET}}
\newcommand{\mnd}[1]{\mathsf{#1}}
\newcommand{\cmd}[1]{\mathsf{#1}}

\newcommand{\Coalg}[1]{\mathsf{#1}\text-\cat{Coalg}}
\newcommand{\Alg}[1]{\cat{#1}\text-\cat{Alg}}
%EH
\newcommand{\Sig}{\mathbf \Sigma\text-\cat{str}}
\newcommand{\Kl}[1]{\cat{Kl}(\mathsf{#1})}
\newcommand{\aone}{{\mathbf 1}}
\newcommand{\atwo}{{\mathbf 2}}
\newcommand{\athree}{{\mathbf 3}}
\newcommand{\Rad}{\cat{Rad}\text-\cat{Cat}}
%\newcommand{\mb}{{\mathbf {#1}}}

\renewcommand{\vec}{\boldsymbol}

%Eli's macros
\newcommand{\azero}{{\mathbf 0}}
\renewcommand{\epsilon}{\varepsilon}
\newcommand{\up}{\mathop{\uparrow}}
\newcommand{\down}{\mathop{\downarrow}}
\newcommand{\diff}{\setminus}
\renewcommand{\prec}{\preccurlyeq}
\renewcommand{\geq}{\geqslant}
\renewcommand{\leq}{\leqslant}
\renewcommand{\nleq}{\nleqslant}
\renewcommand{\ngeq}{\ngeqslant}
\renewcommand{\emptyset}{\varnothing}
\DeclareMathOperator{\coker}{coker}

%---
%\newcommand{\DP}{\mathbb{D}}
%\renewcommand{\P}{\mathbb{P}}
%\newcommand{\FP}{\mathbb{FP}}
%\newcommand{\FL}{\mathbb{FL}}
%\DeclareMathOperator{\FP}{\mathsf{\bf FP}}
\DeclareMathOperator{\FP}{\bf FP}
\DeclareMathOperator{\FL}{\bf FL}


\DeclareMathOperator{\Pol}{\mathsf{Pol}}
\DeclareMathOperator{\Inv}{\mathsf{Inv}}
%Class operators
\DeclareMathOperator{\Hs}{\mathsf{H}}
\DeclareMathOperator{\Iss}{\mathsf{I}}
\DeclareMathOperator{\Ss}{\mathsf{S}}
\DeclareMathOperator{\Si}{Si}
\DeclareMathOperator{\Ps}{\mathsf{P}}
\DeclareMathOperator{\PU}{\mathsf{P_U}}
\DeclareMathOperator{\ISP}{\mathsf{ISP}}
\DeclareMathOperator{\ISrP}{\mathsf{I}\mathsf{S}_{\mathrm{r}}\mathsf{P}}
\DeclareMathOperator{\IDSP}{\mathsf{IDSP}}
\DeclareMathOperator{\HSP}{\mathsf{HSP}}
\DeclareMathOperator{\Sub}{Sub}
\DeclareMathOperator{\dom}{dom}
\DeclareMathOperator{\gr}{gr}
%\DeclareMathOperator{\gr}{gr}


\newcommand{\yo}{\text{\usefont{U}{min}{m}{n}\symbol{'210}}}
\renewcommand{\emptyset}{\varnothing}

\newcommand{\richard}[1]{{\red #1}}
%

\begin{document}
\leftmargini=2em 
\title{Clones, coclones, and the duality between relations and operations}
\author{Eli Hazel} 
\address{School of Mathematical and Physical Sciences, Macquarie University, NSW 2109, Australia} 
\email{eli.hazel@mq.edu.au}

\subjclass[2000]{Primary: }
\date{\today}

\thanks{The support of a Macquarie University Research Excellence
  Scholarship is gratefully acknowledged.}

\begin{abstract}
  The equivalence between Lawvere theories in category theory and
  abstract clones in universal algebra is well-known.
  The notion of relational clone is less so, but deserves more attention. These were developed in general algebra as a tool to study concrete
  clones due to a Galois connection between relations and
  operations. Concrete clones form the fix-points on the side of
  operations and relational clones form the fix-points on the side of relations.
  The lattices of fix-points are dually isomorphic --- a duality between
  clones of operations and clones of relations, or coclones. 
A category theoretic treatment linking this duality with finite product (Lawvere) theories has been
considered an open problem. In this work, we give such a category theoretic
treatment, leading to a dual equivalence
  between categories whose objects on each side are certain classes of
  finite limit theories.  
\end{abstract}
\maketitle


\section{Introduction}
\label{sec:introduction}
Concrete clones have been objects of study since as far back as the
1920s, when Emil Post gave a description of all clones concrete on the
2-element set~\cite{Kerkhoff:2014aa}. However, the abstract notion was a later development
originating as Lawvere's algebraic theories in~\cite{Lawvere:1963aa}. General algebraists adopted algebraic theories in the form
of \emph{abstract clones}, finding use in the presentation of algebraic
theories abstracted from signature, while concrete
clones provide a complete representation of algebraic structure on
a given base set. Building on this, algebraists developed the
notion of a \emph{relational clone}, or \emph{coclone},
which arise as the fixpoints on the relation side of a Galois
connection between relations and operations on a given set. The name coclone suggests a duality, and indeed the fixpoints on the operation
side are clones, and the clones and coclones on a given set form dually
isomorphic lattices. Known commonly as the \(\Pol\)-\(\Inv\) Galois
connection, it was described by
McKenzie {\it et. al.} in~\cite{McKenzie:1987aa}
%in their monograph \emph{Algebras, Lattices,
%Varieties I}
as `the most
basic Galois connection in algebra'. 

The \(\Pol\)-\(\Inv\) Galois
connection was the focus of the 2013 Workshop on Algebra, Coalgebra and Topology at the
University of Bath, which was organised
as a joint venture between algebraists and categorists.
%to encourage dialogue between the disciplines.
In the preface to the workshop's Proceedings,
%of the Workshop on Algebra, Coalgebra and Topology (WACT 2013),
editors John Power and Cai Wingfield invite a category theoretic treatment
of the connection --- and thus presumably the duality --- between clones and coclones in terms of Lawvere
theories.  This work provides an answer to that invitation.
However, our interest in this problem originates in its connection
with a certain class of concrete dualities, the `natural dualities' of
\cite{Clark:1998aa}; a connection we will make clear in forthcoming
work. We note that the present work owes much to \cite{Clark:1998aa}.

Returning to the matter at hand, it
turns
out that to capture \(\Pol\)-\(\Inv\) duality, we need more than the
Lawvere theory alone can provide, in general. Nonetheless, we obtain a dual equivalence between categories
whose objects on each side are certain classes of \emph{finite limit
  theories}, while the finite product (Lawvere) theories are
recoverable as full subcategories thereof. 
We formulate the duality in terms of what we refer to as \emph{abstract
  coclones}, which we introduce in the lead up to describing the
duality in Section~\ref{sec:quasivariety}. The description of their
dual theories, also in Section~\ref{sec:quasivariety}, is based on a characterisation
of finite limit
theories of quasivarieties given in \cite{Adamek:1998aa}. The
categories involved, along with the duality itself, is the content of Section~\ref{sec:duality}.


\section{Background}
\label{sec:background}



\subsection{Model theory}
\label{sec:logic-background}
%.
This section is based on~\cite{Adamek:1994aa}.
%.
\begin{Defn}[Single-sorted signature \(\Sigma\)]
  \label{def:1-signature}
  A (single-sorted, finitary) \emph{signature} \(\Sigma\) consists of:
  \begin{enumerate}[(i)]
  \item Disjoint sets \(\Sigma_{\O}\) of \emph{operation symbols} and
    \(\Sigma_{\R}\) of \emph{relation symbols};
  \item A function
    \(\mathsf{ar} \colon \Sigma_{\O}\rightarrow \mathbb{N}\)
    assigning each operation symbol \(\tau\) in \(\Sigma_{\O}\) its arity
    \(\tau \colon s^{n} \rightarrow s\), where
    \(s\) is the sort and \(n\in \mathbb{N}\);
  \item A function
    \(\mathsf{ar} \colon \Sigma_{\R}\rightarrow \mathbb{N}\)
    assigning each relation symbol its arity
    \(\rho \colon s^{k}\), where
    \(s\) is the sort and \(k \in \mathbb{N} \).
  \end{enumerate}
\end{Defn}

\begin{Defn}[Structure for signature \(\Sigma\)]
  \label{def:sig-str}
  A \emph{\(\Sigma\)-structure} \(\mathsf{V} \defeq \langle V; \Sigma
  \rangle\) of signature \(\Sigma\) is a set \(V\) equipped with with an interpretation of each
  \(n\)-ary operation symbol \(\tau \in \Sigma_{\O}\) as a function \(\dbr{\tau}
  \colon V^{n} \rightarrow V\),
  and an interpretation of each \(k\)-ary relation symbol \(\rho \in \Sigma_{\R}\) as a
  subset \(\dbr{\rho} \subseteq V^{k}\).
\end{Defn}

\begin{Defn}[Category of \(\Sigma\)-structures]
  \label{def:cat-sig-st}
  The \emph{category of \(\Sigma\)-structures} over a signature
  \(\Sigma\) has
  \begin{itemize}
  \item \emph{Objects} are \(\Sigma\)-structures;
  \item \emph{Morphisms} \(h \colon \mathsf{A} \rightarrow \mathsf{B}\) are
    functions which commute with the operations and preserve the
    relations. Explicitly:
    
    For each \(n\)-ary operation symbol \(\tau
    \in \Sigma_{\O }\) and \(a_{1},\dots, a_{n}\in A\), 
    \begin{equation*}
      h(\dbr{\tau}_{\mathsf{A}}(a_{1},\dotsc,a_{n})) = \dbr{\tau}_{\mathsf{B}} (h(a_{1}),\dotsc , h(a_{n})),
    \end{equation*}
    and for each \(n\)-ary relation symbol \(\rho \in \Sigma_{\R}\),
    \begin{equation*}
      (a_{1},\dotsc, a_{n}) \in \dbr{\rho}_{\mathsf{A}} \implies \left(h (a_{1}), \dotsc ,h(a_{n})\right) \in \dbr{\rho}_{\mathsf{B}} \rlap{ .}
    \end{equation*}
  \end{itemize}
\end{Defn}


There are a number of useful facts about \(\Sig\) which can be found
in~\cite{Adamek:1994aa}. 
\begin{enumerate}[label=(\alph*), ref=(\alph*)]
\item
  \label{it:sig-lfp}
  \(\Sig\) is locally finitely presentable, and so complete and
  cocomplete;
\item
  \label{it:sig-factor-sys}
  \(\Sig\) has (epi, regular mono) and (regular epi, mono)
  factorisations of morphisms;
\item
  \label{it:sig-forgetful}
  There is a natural forgetful functor
  %from the category of \(\Sigma\)-structures to \(\Set\): 
  \(\mathrm{G}
  \colon \Sig \rightarrow \Set\) which sends a \(\Sigma\)-structure
  its underlying set and a morphism of \(\Sigma\)-structures to its
  underlying function;
\item
  \label{it:sig-free}
  The forgetful functor has a
  left adjoint \(\mathrm{F}
  \colon \Set \rightarrow \Sig\), which sends a set \(A\) to the
  structure whose underlying algebra is freely generated by \(A\) over
  \(\Sigma_{\O}\), and with all relations empty.
  %\item
  %  \label{it:sig-fp-objects}
  %  An object in \(\Sig\) is finitely presentable just when 

\end{enumerate}


\begin{Rk}[Subalgebras vs substructures]
  \label{rk:subalg-vs-substr}
  An important distinction must be noted here. In the sequel, we will
  have occasion to consider structures for signatures which are purely operational --- containing only
  operation symbols ---
  and this will mean a distinction must be drawn between a \emph{subalgebra} and
  \emph{substructure}. If a signature \(\Omega\) is purely
  operational the objects of \(\cat{\Omega St}\) will be referred to as
  \(\Omega\)-algebras. In \(\cat{\Omega St}\), the inclusion of a subalgebra \(A
  \hookrightarrow B\) is a monomorphism. Moreover, for any morphism
  \(h \colon A \rightarrow B\) of \(\Omega\)-algebras, we
  have the \emph{image factorisation}
  \begin{equation}
    \label{eq:image-factor-alg}
    \cd{
      {A} \ar[rr]^-{h}
      \ar@{->>}[dr]_-{e} & &
      {B}  \\
      &
      \ar@{^{(}->}[ur]_-{m}
      {h_{!}(A)}
    }
  \end{equation}
  given by the (regular epi, mono) factorisation of \(h\), with 
  \(h_{!}(A)\) a subalgebra of \(B\).

  On the
  other hand, if a signature \(\Psi\) has relation symbols, the image
  factorisation of a morphism \(f \colon
  \mathsf{X} \rightarrow \mathsf{Y}\) in \(\cat{\Psi St}\) does \emph{not}
  necessarily give a substructure of \(\mathsf{Y}\). 
  However, the inclusion of a substructure is a
  \emph{regular} monomorphism, and indeed
  %\(f \colon
  %\mathsf{X} \rightarrow \mathsf{Y}\) of , we have
  the \emph{regular} image factorisation
  \begin{equation}
    \label{eq:reg-image-factor}
    \cd{
      {\mathsf{X}} \ar[rr]^-{f}
      \ar[dr]_-{e} & &
      {\mathsf{Y}}  \\
      &
      \ar@{ >->}[ur]_-{m}
      {f_{!}(\mathsf{X})}
    }
  \end{equation}
   given by the (epi,
  regular mono) factorisation of \(f\) does give a substructure
  \(f_{!}(\mathsf{X})\) of \(\mathsf{Y}\).
\end{Rk}

%\subsection{Finite limit theories and Gabriel-Ulmer duality}
%\label{sec:FLTs}

\subsection{Clones of operations and clones of relations}
\label{sec:clones-rel-clones}

Let \(\cat{\Delta_{0}}\) be a skeleton of the category of linearly
ordered finite
sets and consider its inclusion \([\thg] \colon
\cat{\Delta_{0}} \rightarrow \cat{Set}_{f}\) where \(\cat{Set}_{f}\) denotes
the category of finite sets. We regard \(A^{n}\) as notation for the set \(\Set([n],A)\).

\begin{Defn}
  \label{def:conc-clo}
  %A set \(\mathscr{C} \subseteq \O_{A}\) is
  A \emph{concrete clone}, or  \emph{clone of operations} \(\mathscr{C}\) on a set \(A\) consists
  of:
  %an indexed family of sets 
  %indexed family \((\mathscr{C}(n))_{n\in \cat{\Delta_{0}}
  %}\), where:
  \begin{itemize}
  \item For each \(n\in \mathbb{N}
    \), a set \(\mathscr{C}(n) \subseteq \Set(A^{n},A)
    \) of \emph{\(n\)-ary operations};
  \item \(\mathscr{C}(n)\) contains the projections \(\pi_{i}^{n} \colon A^{n}
    \rightarrow A \);

  \item \(\mathscr{C}\) is closed under \emph{composition of operations} in the sense that
    %for all \(m,n\in\mathbb{N}\),
    for any \(f \in \mathscr{C}(n)
    \) and \(n\)-tuple \((g_{1},\dotsc,g_{n}) \in \mathscr{C}(k)^{n}
    \), there is a \(k\)-ary
    operation \(f(g_{1},\dots,g_{n})\)
    defined on each \((a_{1},\dotsc,a_{k})\in A^{k}\) by
    \begin{equation}
      \label{eq:15}
      f(g_{1},\dotsc,g_{n})(a_{1},\dotsc,a_{k}) = f(g_{1}(a_{1},\dotsc,a_{k}),\dotsc,g_{n}(a_{1},\dotsc,a_{k})).
    \end{equation}
  \end{itemize}
\end{Defn}

\begin{Ex}[Full concrete clone]
  \label{ex:full-cc}
  Given a set \(A\), take \(\mathscr{O}_{A}(n)\) to be the entirety of 
  \(\Set(A^{n},A)\) for each \(n\). 
  %\begin{equation*}
  %  \mathscr{O}_{A}(n)\defeq \Set(A^{n},A).
  %\end{equation*}
  The set \(\mathscr{O}_{A}\) is known as the
  \emph{full clone of operations}
  on \(A\). 
\end{Ex}




\begin{Defn}
  \label{def:all-rels-on-set}
  Let \(A\) be a non-empty set. Considering an \(n\)-ary relation on \(A\) as a
  subset of \(A^{n}\), the set of all finitary relations on~\(A\) is
  given by
  \begin{equation*}
    \R_{A}\defeq \bigcup\limits_{n\in \mathbb{N}}
    \Sub(A^{n})
  \end{equation*}
  where \(\Sub(A^{n})\) denotes the set of subsets of \(A^{n}\).
\end{Defn}


\begin{Defn}
  \label{def:compatibility-matrix}
  Given an operation \(f \colon A^{n} \rightarrow A\), we have \(f^{k}
  \colon (A^{n})^{k} \rightarrow A^{k}\), with
  \begin{equation*}
    f^{k}
    \begin{pmatrix}
      a_{11} & a_{12} & \cdots & a_{1n} \\
      a_{21} & a_{22} & \cdots & a_{2n} \\
      \vdots  & \vdots  & \ddots & \vdots  \\
      a_{k1} & a_{k2} & \cdots & a_{kn} 
    \end{pmatrix}
    =
    \begin{pmatrix}
      f(a_{11}, \dotsc , a_{1n}) \\
      f(a_{21}, \dotsc , a_{2n}) \\
      \vdots  \\
      f(a_{k1}, \dotsc , a_{kn})
    \end{pmatrix}\rlap{ .}
  \end{equation*}
  The operation \(f\) is said to \emph{preserve}
  %or be \emph{compatible} with
  a relation \(S \subseteq A^{k}\) if for each \(k\times n\) matrix over \(A\),
  \begin{equation*}
    \begin{pmatrix}
      a_{1j} \\
      a_{2j} \\
      \vdots \\
      a_{kj}
    \end{pmatrix} \in S  \quad (\forall j\in n)
    \implies
    \begin{pmatrix}
      f(a_{11}, \dotsc , a_{1n}) \\
      f(a_{21}, \dotsc , a_{2n}) \\
      \vdots  \\
      f(a_{k1}, \dotsc , a_{kn})
    \end{pmatrix} \in S  \rlap{ .}
  \end{equation*}
\end{Defn}

\begin{Lemma}
  \label{lem:compatibility}
  Given an operation \(f \colon A^{n} \rightarrow A\) and a relation \(S
  \subseteq A^{k}\), the following are equivalent:
  \begin{enumerate}
    [label=(\alph*), ref=(\alph*)]
  \item
    \label{it:preservation-of-rel}
    \(f\) preserves \(S \);
  \item
    \label{it:closed-under-op}
    \(S \) is \emph{closed} under \(f\), that is, \(S\) is a subalgebra of
    \(\langle A; f \rangle^{k}\);
  \item
    \label{it:morphism-rel-str}
    \(f\) is a morphism of relational structures
    \(\langle A; S  \rangle^{n} \rightarrow \langle A; S \rangle\).
  \end{enumerate}
\end{Lemma}

\begin{proof}
  Let \(\iota \colon S \rightarrow
  A^{k}\) be the inclusion of \(k\)-ary
  relation \(S\) and let \(f^{k}\!\upharpoonright\) denote the evident
  restriction of \(f^{k}\). Then each of Items~\ref{it:preservation-of-rel},~\ref{it:closed-under-op},
  and~\ref{it:morphism-rel-str} may be interpreted as requiring that the diagram
  \begin{equation*}
    \cd{
      {S^{n}} \ar[r]^-{f^{k}\upharpoonright}
      \ar[d]_{\iota^{n}} &
      {S} \ar[d]^{\iota} \\
      {(A^{k})^{n}} \ar[r]_-{f^{k}} &
      {A^{k}}
    } 
  \end{equation*}
  commutes.
\end{proof}


\begin{Defn}
  \label{def:3}
  We say a relation and an operation are \emph{compatible} if any of the
  items in Lemma~\ref{lem:compatibility} hold (and so all of them do).
\end{Defn}


\begin{Rk}
  \label{rk:es-nullaries}
  Algebraists have traditionally opted not to allow an algebra to have
  empty underlying set, and to exclude nullary
  operations from clones of operations. However, without their inclusion, the categories we define
  later will lack necessary (finite) limits and colimits. We note that~\cite{Behrisch:2014aa} does give a formulation of
  compatibility of operations and relations, but their approach is
  somewhat different to ours.
 
  
  An empty \(k\)-ary relation \(\emptyset{}^{k}\)
  %\(r \colon S \rightarrow A^{k}\)
  is closed under \(u \colon A^{n} \rightarrow A\)  if the following
  diagram commutes:
  \begin{equation*}
    \cd{
      {\emptyset{}^{k\times n}} \ar[r]^-{u^{k}\upharpoonright}
      \ar[d]_{!^{n}} &
      {\emptyset{}^{k}} \ar[d]^{!} \\
      {(A^{k})^{n}} \ar[r]_-{u^{k}} &
      {A^{k}}
    } \rlap{ .}
  \end{equation*}
  If \(n \neq 0\), the map at the top of the
  square is simply the identity map \(1_{\emptyset{}^{k}}\), so to ask
  that a non-nullary operation \(u\) is compatible with an empty
  relation is no condition at all on \(u\).
  But if \(u\) is nullary, we have that \(\emptyset{}^{k\times 0} = \{1_{\emptyset{}^{k}}\}\), so the
  map at the top of the square does {\it not exist}.

  Thus an empty relation cannot be compatible with any nullary
  operation but provides no constraint on non-nullary operations, which is what we
  want.
\end{Rk}

\begin{Rk}
  \label{rk:finite-vs-non}
  The definition of a relational clone on a set \(A\) is usually given
  in terms of
  being a fix-point of the \(\Pol\)-\(\Inv\) Galois connection. If \(A\) is not finite, this requires a notion of
  \emph{local closure}; this ensures that the lattice of clones of
  operations and the lattice of clones of relations are dually isomorphic. 

  However, for reasons which will become clear, up until
  Section~\ref{sec:duality} we work only with relations and operations on
  finite sets. Giving the more general version would introduce
  unnecessary complications. We refer the interested reader
  to~\cite{Kerkhoff:2014aa} for an overview and a wealth of references.
\end{Rk}



\subsection{The Pol-Inv Galois connection and the duality between
  clones of operations and clones of relations}
\label{sec:pol-inv-galois}

The notion of `compatibility' between operations and relations given
in Definition~\ref{def:3} induces a binary relation
\(\C \subseteq \O_{A} \times \R_{A}
\).
As with any binary relation, \(\C\) induces a Galois connection
between powersets. Known as
the \(\Pol\)-\(\Inv\) Galois connection\footnote{The names of the adjoints come from the traditional terminology of a
  relation being an `invariant' of a set of operations and an
  operation being a `polymorphism' of a set of relations~\cite{Kerkhoff:2014aa}. These terms
  are no longer used in this context. Nonetheless, they continue to lend their names to the adjoints. 
}, it has adjoints
\begin{equation*}
  \xymatrixcolsep{3.5pc}
  \xymatrixrowsep{2.5pc}
  \cd{
    {\mathscr{P} (\O_{A})^\mathrm{op}} \ar@<-4.5pt>[r]_-{\Inv} \ar@{<-}@<4.5pt>[r]^-{\Pol} \ar@{}[r]|-{\bot} &
    {\mathscr{P} (\R_{A})
      \rlap{ .}} 
  }
\end{equation*}

\begin{Prop}
  \label{prop:3}
  Given a finite
  set
  \(A\), the fixpoints \(\Pol\Inv(F)=F\) are precisely the clones of
  operations on \(A\) and the fixed points \(\Inv\Pol(R)=R\) are
  precisely the clones of relations on \(A\). Moreover, the lattices
  of fixpoints are dually isomorphic.  
\end{Prop}

Using compatibility from Definition~\ref{def:3}, we have that the mappings 
\begin{equation*}
  \Inv \colon \mathscr{P} (\O_{A})^{\mathrm{op}}  \rightarrow \mathscr{P} (\R_{A}) \quad \text{ and } \quad \Pol \colon \mathscr{P} (\R_{A})^{\mathrm{op}}
  \rightarrow \mathscr{P} 
  (\O_{A})
\end{equation*}
may be given by 
\begin{equation*}
  \Inv(F) \defeq \{S \in \R_{A} : S  \text{ is compatible with every } \tau \in F\} 
\end{equation*}
and
\begin{equation*}
  \Pol(R) \defeq \{ \tau \in \O_{A} : \tau \text{ is compatible with every } S \in R \}.
\end{equation*}

Using the characterisations of compatibility in Lemma~\ref{lem:compatibility}, we obtain:
\begin{Cor}
  \label{cor:compatibility-clone-coclone}
  Assume \(\mathscr{R}
  = \Inv (\mathscr{T}
  )\) and \(\mathscr{T}
  = \Pol (\mathscr{R}
  )\)
  are clones of relations and operations respectively on a set \(A\).
  The relations in \(\mathscr{R}
  \) are precisely those subsets which
  correspond to subalgebras of finite powers of \(A\) under the algebra
  structure given by \(\mathscr{T}
  \) on \(A\) and extended pointwise.
\end{Cor}


\begin{Defn}[Term-closed sets of relations]
  \label{def:term-closure}
  A set of relations \(\mathscr{R}
  \subseteq \R_{V}\) is called \emph{term-closed}
  when it is closed with respect to the following:
  
  \begin{enumerate}
    [label=(\roman*), ref= (\roman*)]
  \item
    \label{it:trivial}
    \emph{Trivial relations}:
    A \emph{trivial}, or \emph{generalised diagonal} relation on~\(V\)
    is one induced by an equivalence relation on indices in the following
    way: Given an equivalence relation \(\theta\) on \([k]\), there is
    a \(k\)-ary relation on \(V\) given by
    \begin{equation*}
      \Delta_{\theta} = \{(x_{1}, \dots x_{k}) \in V^{k} : (\forall (i,j) \in \theta)\;\; x_{i} = x_{j}\}.
    \end{equation*}
    
  \item
    \label{it:projection}


    \emph{Projections}:
    %\label{def:ppf-proj}
    Let \(\iota \colon k \rightarrow n\) be a monomorphism in
    \(\cat{\Delta_{0}}\).
    %Define \(\bar{\phi}(x_{1}, \dots x_{n}) =
    %(x_{\phi(1)}, \dots x_{\phi(k)}) \)
    Given an \(n\)-ary relation \(R\), the \(k\)-ary \emph{projection}
    of \(R\) along \([\iota]\) is given by
    \begin{equation*}
      \iota(R)  \defeq \{ (x_{\iota(1)}, \dots , x_{\iota(k)}) \in V^{k} :
      (x_{1}, \dots x_{n}) \in R \}
    \end{equation*}

  \item
    \label{it:concat}
    \emph{Concatenation}:
    %\label{def:ppf-prod}
    Given an~\(m\)-ary relation~\(R\) and a~\(k\)-ary relation~\(S\),
    their concatenation \(RS\) is the \((m+k)\)-ary relation given by their cartesian product:
    \begin{flalign*}
      %R \times S = 
      \{ (x_{1}, \dots x_{m}, x_{m+1}, \dots x_{m+k}) \in V^{m+k} :
      (x_{1}, \dots x_{m}) \in R,
      \; 
      %&& \\
      %& \text{ and } 
      (x_{m+1}, \dots x_{m+k}) \in S \} \rlap{.} &
    \end{flalign*}

  \item
    \label{it:int1}
    \emph{Intersection}:
    %\label{def:ppf-int}
    Given~\(n\)-ary relations~\(R\) and~\(S\), their
    intersection is given by
    \begin{equation*}
      R \cap S = \{ (x_{1}, \dots x_{n}) \in V^{n} : (x_{1}, \dots x_{n}) \in R  \text{ and }  (x_{1}, \dots x_{n}) \in S \} \rlap{.}
    \end{equation*}

  \item
    \label{it:permut}
    \emph{Coordinate permutation}:
    %\label{def:ppf-permut}
    A permutation \(s \colon [n]
    \rightarrow [n]\)
    induces a permutation of variables on an \(n\)-ary relation \(R\), giving a
    relation
    \begin{equation*}
      s[R] =  \{ (x_{s(1)},\ldots,x_{s(n)}) \in V^{n} : (x_{1}, \ldots , x_{n})\in R \}
    \end{equation*}
  \end{enumerate}
\end{Defn}

\begin{Rk}
  \label{rk:def:term-closure}
  The items in Definition~\ref{def:term-closure} are essentially the
  same as those given
  on p.11 of~\cite{Freese:2022aa} without providing a name. We used different notation to
  enforce differentiating between
  categorical product and the concatenation of relations (which is
  given by the binary product of the subsets). The terminology
  term-closed is used in
  \cite{Clark:1998aa} for something different, though related. We use
  it due to the relationship with closure under terms derived from
  positive-primitive formulae.
  %{\blue (elucidated on in Section~\ref{sec:primitive-pos})}.
\end{Rk}


The following is Theorem~9.14 (ii)  in~\cite{Freese:2022aa}.
\begin{Thm}
  \label{thm:closure-coclone}
  Given \(R \subseteq \R_{V}\), we have that \(R = \Inv\Pol(R)\) (and
  is thus a relational clone on \(V\)) if and only if \(R\) is term-closed,
  that is, contains all the
  trivial relations and is closed under intersection of relations of
  the same arity, relations induced by permutation of variables,
  concatenation, and projection. 
  %\begin{enumerate}[(a)]
  %\item Product;
  %\item Intersection;    
  %\item Projection (restriction would be a better term);
  %\item Coordinate permutations; and
  %\item Contains all trivial relations.
  %\end{enumerate}
\end{Thm}

\begin{Rk}
  \label{rk:15}
  We will shift between regarding an element of \(V^{k}\)
  as a function \(\alpha \colon [k] \rightarrow V\) or as a
  \(k\)-tuple of elements of \(V\) depending on what we find
  convenient for the matter at hand.
\end{Rk}


%\subsection{Relational clones and primitive-positive formulae}
%\label{sec:primitive-pos}
%
%{\blue
%  How much to say about this should be considered.
%  \begin{Defn}
%    \label{def:ppf}
%    A \emph{positive-primitive} formula in a first order language \(\mathscr{L} 
%    \)...
%  \end{Defn}
%}
%\subsection{Closure under primitive-positive formulae}
%\label{sec:clos-under-prim}

\subsection{Reflective subcategories}
\label{sec:reflective-subcats}

Assume we have an adjunction
\begin{equation}
  \label{eq:adj-GF}
  \cd{
    {\cat{D}} \ar@<-4.5pt>[r]_-{G} \ar@{<-}@<4.5pt>[r]^-{F} \ar@{}[r]|-{\bot} &
    {\cat{C}}   
  }
\end{equation}
with unit \(\eta \colon 1_{\cat{C}} \rightarrow GF\) and counit
\(\epsilon \colon FG \rightarrow 1_{\cat{D}}\). 

\begin{Lemma}
  \label{lem:subcat-unit-mono-closure}
  Let \(\cat{A}\) be the full subcategory of \(\cat{C}\) whose
  objects are \(C\) such that the unit
  \(\eta_{C} \colon C \rightarrow GFC\)
  is a monomorphism. If there exists a monomorphism \(m \colon C
  \rightarrow GD\), then \(C \in \cat{A}\).
\end{Lemma}

\begin{proof}
  %Notice that if \(m\) is mono and \(u \cdot v \)
  Assume we have the situation as stated above, and morphisms \(f,g
  \colon B \rightarrow C\) such that \(\eta_{C}f = \eta_{C}g\). By adjointness, there
  exists a unique morphism \(\tilde{m} \colon FC \rightarrow D\) such that
  \(G\tilde{m}\cdot\eta_{C} = m\), so \(G\tilde{m}\cdot\eta_{C} \cdot f = m
  f = G\tilde{m}\cdot\eta_{C} \cdot g = m g\). As \(m\) is monic, \(m f =
  m g\), so \(f = g\).

  Thence \(\eta_{C}\) is a monomorphism and so \(C \in \cat{A}\).
  %Now, \(G\tilde{m}\eta_{C} f = G\tilde{m}\eta_{C} g\)
\end{proof}


\begin{Prop}
  \label{prop:orthog-reflective-subcat}
  Assume \(\cat{C}\) has a (regular epi, mono) factorisation system
  %, that is, every morphism in \(\cat{C}\) factors as a (e, m),
  and \(\cat{A}\) is closed under the monomorphisms 
  in the manner described in
  Lemma~\ref{lem:subcat-unit-mono-closure}.
  Then \(\cat{A}\) is a regular epi-reflective
  subcategory of \(\cat{C}\), with reflector given by the regular epi
  component of the (regular epi, mono) factorisation of the
  unit of the adjunction \eqref{eq:adj-GF}.
\end{Prop}

\begin{proof}
  Let \(X\in \cat{C}\) and let
  \begin{equation}
    \label{eq:27}
    \cd{
      {X} \ar[rr]^-{\eta_{X}}
      \ar[dr]_-{e_{X}} & &
      {GFX}  \\
      &
      \ar[ur]_-{m_{X}}
      {\overline{X}}
    }
  \end{equation}
  be the (regular epi, mono)
  %(strong epi, mono)
  factorisation of \(\eta_{X}\).   We will first show that the inclusion of
  \(\cat{A}\) has an adjoint by showing that for each \(X \in
  \cat{C}\), the component \(e_{X}\) satisfies the universal property
  of the unit. That is to say, if \(Y\in \cat{A}\) and there exists \(f \colon X
  \rightarrow Y\), there is a unique extension of \(f\) along \(e_{X}\).
  Consider the following diagram:
  \begin{equation}
    \label{eq:28}
    \cd{
      {X} \ar[rr]^-{f}
      \ar[d]_-{e_{X}} & &
      \ar[d]^-{\eta_{Y}}
      {Y}  \\
      {\overline{X}} \ar[r]_-{m_{X}}
      \ar@{-->}[urr]^-{}
      &
      {GFX} \ar[r]_-{GFf} & {GFY}
    }\rlap{ .}
  \end{equation}
  Note that the outer rectangle commutes by naturality of \(\eta\),
  and that the morphism indicated by the dotted line exists uniquely
  as
  \(\eta_{Y}\) is in \(\mathscr{M}
  \) and \(e\) is in \(\mathscr{E}
  \) and are thus orthogonal. We now show there is a family of maps
  which form the components of the counit, which must invert each
  \(e_{\overline{X}}\). Observe we have a commuting diagram
  \begin{equation}
    \label{eq:32}
    \cd{
      {\overline{X}} \ar[r]^-{e_{\overline{X}}} \ar[d]_{1} &
      {\overline{\overline{X}}} \ar[d]^{m_{\overline{X}}} \ar@{-->}[dl]_-{p} \\
      {\overline{X}} \ar[r]_-{\eta_{\overline{X}}} \ar[d]_-{e_{\overline{X}}} &
      {GF\overline{X}} \ar[d]^-{1}\\
      {\overline{\overline{X}}} \ar[r]_-{m_{\overline{X}}} & {GF\overline{X}}
    }\rlap{ ,}
  \end{equation}
  where again, \(p\) exists uniquely by the orthogonality of \(\eta_{\overline{X}}
  \) and \(e_{\overline{X}}\in 
  \). So
  \(p e_{\overline{X}} = 1_{\overline{X}}\), and
  \(m_{\overline{X}} e_{\overline{X}} p = m_{\overline{X}}\) and as \(m_{\overline{X}}\) is mono,
  \(e_{\overline{X}} p = 
  1_{\overline{X}}\). 

  Thus, the inclusion of \(\cat{A}\) into
  \(\cat{C}\) has an adjoint, with unit \(e_{X} \colon X \rightarrow
  \overline{X}\) consisting of the regular epi component of
  the factorisation of \(\eta_{X}\). By construction, each \(e_{X}\)
  is regular epi and so indeed, \(\cat{A}\) is a regular epi-reflective subcategory of \(\cat{C}\) as claimed.
\end{proof}

There are some properties of reflective subcategories we will find
useful:

\begin{enumerate}[label=(\roman*), ref= (\roman*)]
\item
  \label{it:reflection-monadic}
  A full subcategory is reflective if and only if its inclusion is monadic
  (and thus must have a left
  adjoint);
\item
  \label{it:reflection-idempotent}
  In particular, its
  induced monad is idempotent;

\item
  \label{it:reflection-lims}
  The inclusion functor creates all limits that exist in its
  codomain category;
\item
  \label{it:reflection-colims}
  The domain category inherits colimits from the codomain
  category by precomposing with the reflector (unit);
\item
  \label{it:reflection-co-completeness}
  If the codomain category is closed under some class of colimits, so
  is the domain category. 
\end{enumerate}

\section{Syntactic categories and functorial semantics}
\label{sec:via-syntactic-cats}

\subsection{Syntactic category of a concrete clone}
\label{sec:clone-cat}

\begin{Defn}[Syntactic category of a concrete clone]
  \label{def:cc-abstract-clone}
  Let \(\mathscr{T} 
  \subseteq \O_{V}\) be a concrete clone. We define a
  subcategory \(\cat{T}\) of $\cat{Set}_f$ category \(\cat{T}\) from \(\mathscr{T} 
  \) as follows:
  \begin{itemize}
  \item \emph{Objects} $[n,V]$ are sets of the form $V^n$ for some $n
    \in \mathbb{N}$;
  \item A \emph{morphism} $g \colon [n,V] \rightarrow [m,V]$ is a
    function $V^n \rightarrow V^m$ for which each of the projections $\pi_1 g, \dots, \pi_m g \colon V^n
    \rightarrow V$ is in $\mathscr{T}(n)$.
  \item \emph{Composition} of morphisms is given simply by composition of functions.
  \end{itemize}
\end{Defn}

\begin{Prop}
  \label{prop:cc-cat}
  The category $\cat{T}$ of the preceding definition is 
  closed under finite products in $\cat{Set}_f$, and hence is
  equivalent to a
  Lawvere theory. The faithful inclusion functor $\M \colon \cat{T}
  \rightarrow \cat{Set}$ exhibits $V$ as a $\cat{T}$-algebra.
\end{Prop}


\subsection{Syntactic category of a relational clone}
\label{sec:coclone-to-clone}

In this section, assume \(\mathscr{R} 
\) is a relational clone on a
(finite) set \(V\), that
is, \(\mathscr{R} \subseteq \R_{V}
\) and \(\Inv\Pol(\mathscr{R}
) = \mathscr{R}
\).  As clones of operations and clones of relations on \(V\) are in bijection,
there is a unique clone \(\mathscr{T}
\subseteq \O_{V}
\) such that \(\Pol(\mathscr{R}
) = \mathscr{T}
\) and \(\Inv(\mathscr{T}
) = \mathscr{R}
\). Also assume that \(\cat{T}\) is the syntactic category (finite product
theory) constructed from \(\mathscr{T}
\) as described in Subsection~\ref{sec:clone-cat}. 

\begin{Defn}[Syntactic category from a relational clone]
  \label{def:syn-cat-cc}
  Given \(\mathscr{R}\), we define the category \(\cat{S}\) as follows:
  \begin{itemize}
  \item \emph{Objects} are pairs $(k, S)$ where $k \in
    \mathbb{N}$ and $S \in \mathscr{R}(k)$;

  \item \emph{Morphisms} $f \colon (k,S)
    \rightarrow (n,T)$ is a function $f \colon S \rightarrow T$ for
    which the graph \[\mathrm{gr}(f) \defeq \{(x, y) \in V^{k} \times V^n : x \in S,
      y \in T \text{ and } f(x) = y\}\] is in $\mathscr{R}(k+n)$.
    
  \item \emph{Composition of morphisms} are given by
    composition of functions; identity morphisms are given by
    identity functions.
  \end{itemize}
\end{Defn}

The following two lemmas show that this is
indeed well-defined:

\begin{Lemma}
  \label{lem:diag-rel}
  Given any \(k\)-ary relation \(T \in \mathscr{R}(n)\), the diagonal
  relation $\Delta_T = \{(x,x) \in V^n \times V^n : x \in
  T\}$ is in $\mathscr{R}(n+n)$.
\end{Lemma}

\begin{proof}
  We use Theorem~\ref{thm:closure-coclone}: Closure under the trivial
  relations and binary product means that \(\Delta_{V^{n}}\) and \(T T\) are \((n+n)\)-ary relations in
  \(\mathscr{R}\). Then the diagonal relation on \(T\) can be written as
  \begin{equation}
    \label{eq:13}
    \Delta_{T} = T T \cap \Delta_{V^{n}},
  \end{equation}
  which is in \(\mathscr{R}\) as it is closed under intersection of relations of the same arity.  % 
\end{proof}



\begin{Lemma}
  \label{lem:27}
  Assume \(S \in \mathscr{R}(k)\), \(T \in \mathscr{R}(n)\), \(R \in
  \mathscr{R}(m)\), and that the
  graphs \(\gr (g)\) and \(\gr (f)\) of functions \(g \colon S
  \rightarrow T\) and \(f \colon T \rightarrow R\) are relations in
  \(\mathscr{R}(k+n)\) and \(\mathscr{R}(n+m)\) respectively. Then
  \(\gr( f \cdot g)\) is in \(\mathscr{R}(k+m) 
  \).
\end{Lemma}

\begin{proof}
  We use the closure of \(\mathscr{R} 
  \) under concatenation, the diagonal relations, intersection, and
  projection (Theorem~\ref{thm:closure-coclone}). We have that 
  \( (S \Delta_{T}  R) \cap (\gr (g)  \gr (f)) \) is
  in \(\mathscr{R} (k+2n+m)
  \), and is equal to
  \begin{equation*}
    \label{eq:13}
    \{(a,b,b,c) \in V^k \times V^n \times V^n \times V^m : (a,b) \in \mathrm{gr}(g), (b,c) \in \gr(f)
  \end{equation*}
  Projection onto the first and fourth coordinates give the graph of
  \(f\cdot g\).
\end{proof}


\begin{Lemma}
  \label{lem:inclusion-morphisms-cc}
  Let \((k,S)\) and \((k,T)\) be objects in \(\cat{S}\). If
  \(T\subseteq S\), then the inclusion map lifts to a morphism \((k,T) \rightarrow
  (k,S)\) in \(\cat{S}\).
\end{Lemma}

\begin{proof}
  The graph of an inclusion is simply the diagonal of the domain, so this is a corollary of Lemma~\ref{lem:diag-rel}.
\end{proof}

\begin{Lemma}
  \label{lem:gr-functions-powers}
  Let \(g \colon [k] \rightarrow [n]\) be a function. Then
  the graph of \(V^{g} \colon V^{n} \rightarrow V^{k}\) is in \(\mathscr{R}\).
\end{Lemma}

\begin{proof}
  A subset of \(V^{m}\)
  defined only by equality of coordinates corresponds to an
  equivalence relation on \(m\), which by
  Theorem~\ref{thm:closure-coclone} is in \(\mathscr{R}\) as it is
  closed under the generalised diagonal relations. Explicitly, there is an equivalence relation \(\theta\) on \([n]+[k]\) whose equivalence
  classes are given by \(g^{\ast}(j) + \{j\}\) for each \(j\in [n]\). We have
  \begin{flalign*}
    \label{eq:19}
    \gr(V^{g}) & = \{(\alpha, \beta) \in V^{n}\times V^{k} : \beta = \alpha \cdot g \} \\
    & = \{\gamma \in V^{n+k} : \gamma(i) = \gamma(g(i)) \} \\
    & = \{\gamma \in V^{n+k} : (\forall (i,j)\in \theta) \;\gamma(i) = \gamma(j) \} \\
    & = \Delta_{\theta}^{n+k},
  \end{flalign*}
  which is a generalised diagonal relation, and so is in
  \(\mathscr{R}\).
\end{proof}


\begin{Lemma}
  \label{lem:projections-morphisms-cc}
  Let \((k,S)\) and \((n,T)\) be objects in \(\cat{S}\). The graphs of
  the projections \(\pi_{1} \colon T \times S \rightarrow T\) and
  \(\pi_{2} \colon T \times S \rightarrow S\) are in \(\mathscr{R}\).
\end{Lemma}

\begin{proof}
  Let \(i_{n} \colon [n] \rightarrow [n+k]\) be the coprojection map
  in \(\cat{Set}_{f}\). We have that \(V^{i_{n}} \colon V^{n+k}
  \rightarrow V^{n}\) is the same as the evaluation map \(V^{n}\times V^{k}
  \rightarrow V^{n}\), and \(\pi_{1}\) is the restriction of this
  evaluation map to \(T \times S\). Explicit calculations give  
  \begin{flalign*}
    \gr(\pi_{1}) & = \gr(V^{i_{n}}) \cap TST \\
    & = \gr(V^{i_{n}}\upharpoonright_{T\times S \times T}),
    %\{(a,b,c) \in T \times S \times T : a = c\} \\
    %& = \{(\alpha,\beta) \in V^{n+k+n} : \beta = \alpha i_{n}\}
  \end{flalign*}
  which is in \(\mathscr{R}\) by Theorem~\ref{thm:closure-coclone}.
  %Lemma~\ref{lem:gr-functions-powers-restricted}.
  Taking the coprojection \(i_{k}\) instead of \(i_{n}\) gives the
  desired result for \(\pi_{2}\).
\end{proof}


\begin{Lemma}
  \label{lem:factorisation}
  %Let \((k,S)\) and \((n,T)\) be objects in \(\cat{S}\).
  Any morphism \((g,n+k) \colon (n,T) \rightarrow
  (k,S)\) in \(\cat{S}\) admits a factorisation into three morphisms,
  whose underlying functions consist of a bijection,
  followed by an inclusion, followed by a projection, as in the diagram
  \begin{equation}
    \label{eq:3-factorisation}
    %\xymatrixcolsep{4pc}
    %\xymatrixrowsep{3pc}
    \cd{
      {S} \ar[d]_-{(\pi_{1}i)^{-1}} \ar[r]^-{g} & {T} \\
      {\gr(g)} \ar[r]^-{i} &  {S \times T} \ar[u]_-{\pi_{2}}
    }\rlap{ .}
  \end{equation}
\end{Lemma}

\begin{proof}
  Consider the following diagram in \(\cat{Set}_{f}\):
  \begin{equation}
    \label{eq:3-factorisation-2-triangle}
    \xymatrixcolsep{1.4pc}
    \xymatrixrowsep{1.6pc}
    \cd{
      && {S} \ar[ddll]_-{1_{S}} \ar[d]^-{} \ar[ddrr]^-{g} && \\
      && {\gr(g)} \ar[d]^-{i} && \\
      {S} && \ar[ll]_-{\pi_{1}} {S \times T} \ar[rr]^-{\pi_{2}} && {T}
    }\rlap{ .}
  \end{equation}
  Note that the inclusion \(i\) composed with the projection
  \(\pi_{1}\) is a bijection and so has an inverse \((\pi_{1}
  i)^{-1}\), appearing as the unlabelled morphism in the diagram above.
  Also, \(\gr(\pi_{1} i)\) is in \(\mathscr{R}(n+k+k)\) as
  inclusions are in \(\mathscr{R}\) by Lemma~\ref{lem:inclusion-morphisms-cc}, and
  (binary) projections are in \(\mathscr{R}\) by Lemma~\ref{lem:projections-morphisms-cc}.
  Since \(\gr(\pi_{1} i^{-1})\) may be obtained from \(\gr(\pi_{1} i)\) by applying a
  permutation of coordinates, it is in \(\mathscr{R}\) by
  Theorem~\ref{thm:closure-coclone}. It is easily seen that
  diagram~\eqref{eq:3-factorisation-2-triangle} commutes, thus giving the
  underlying diagram of functions of a diagram in
  \(\cat{S}\). The right hand triangle gives the factorisation of \((g,n+k) \colon (n,T) \rightarrow
  (k,S)\) we sought.
\end{proof}

Notice that
for each $(k, S)$ in $\cat{S}$, the set
$S$ has canonical structure of a $\mathscr{T}$-algebra as a subalgebra
of \(\mathscr{T}
\)-algebra $V^k$. 

\begin{Lemma}
  \label{lem:image-is-rel}
  Given objects \((k,S)\) and \((n,T)\) in \(\cat{S}\) and a function
  \(h \colon S \rightarrow T\) which preserves the operations in \(\Pol(\mathscr{R}
  ) = \mathscr{T}
  \), the image of \(h\), denoted \(h_{!}(S)\),
  %, denoted \(h_{!}(S)\),
  is the underlying set of an object \((n,h_{!}(S))\) in \(\cat{S}\). 
\end{Lemma}

\begin{proof}
  %The image of \(h\) is the set \(h_{!}(S)\).
  We have \(h_{!}(S) \subseteq 
  T\subseteq V^{n}\) and \(h_{!}(S)\) is closed under the operations
  of \(\Pol(\mathscr{R}
  )\) on $T$. By Corollary~\ref{cor:compatibility-clone-coclone}, \(h_{!}(S)\) is an
  \(n\)-ary relation in \(\mathscr{R}\).
\end{proof}


\begin{Lemma}
  \label{lem:mor-iff-preserves-ops}
  Given objects \((k,S)\) and \((n,T)\) in \(\cat{S}\), a function
  \(h \colon S \rightarrow T\) is the underlying function of a
  morphism \((k,S) \rightarrow (n,T)\) in \(\cat{S}\) if and only if
  it preserves the operations in \(\Pol(\mathscr{R}
  )\).
\end{Lemma}

\begin{proof}
  Let \((k,S)\) and \((n,T)\) be objects in \(\cat{S}\) and assume the
  graph of the function \(g \colon S \rightarrow T\) is in \(\mathscr{R}\). We will
  use Lemma~\ref{lem:factorisation}.

  
  Recall that Lemma~\ref{lem:compatibility} says that a relation
  \(S \subseteq V^{k}
  \) is
  compatible with an operation on \(V\) just when it is a subset
  closed under that operation, that is, a subalgebra. So an object
  \((k,S)\) 
  in \(\cat{S}\) admits an inclusion \(r \colon S \rightarrow V^{k}\)
  in \(\cat{Set}_{f}\) which preserves the operations in \(\Pol(\mathscr{R}
  )\). For any \(R \subseteq S  \subseteq V^{k}
  \), the inclusion \(R \rightarrow S
  \) commutes with the inclusions
  of \(S\) and \(R\) into \(V^{k}\), and so must also preserve the
  operations in \(\Pol(\mathscr{R}
  )\) (by Lemma~\ref{lem:inclusion-morphisms-cc} all such inclusions are
  morphisms in \(\cat{S}\)).
  
  
  Also, given \(r \colon S \rightarrow V^{k}\) and \(q \colon T
  \rightarrow V^{n}\) are the canonical inclusions, then \(r\times q \colon S
  \times T  \rightarrow V^{k} \times V^{n} \) is also the canonical
  inclusion of the concatenation \(ST\), meaning the algebra structure on
  \(S \times T\) is pointwise. So the projections \(\pi_{1} \colon T
  \times S \rightarrow T\) and 
  \(\pi_{2} \colon T \times S \rightarrow S\) preserve the operations in \(\Pol(\mathscr{R}
  )\).

  Consider diagram~\eqref{eq:3-factorisation-2-triangle}. We have shown that
  the inclusion \(i\) and projections preserve the operations in \(\Pol(\mathscr{R}
  )\), so it remains to show that \(\pi_{1}i^{-1}\) does. But the
  inverse of a function that preserves the operations in \(\Pol(\mathscr{R}
  )\) must also
  preserve the operations in \(\Pol(\mathscr{R}
  )\). We may now conclude that \(g\) does also.
  %{\blue needs some editing, at least.}
  %Every morphism in \(\cat{S}\) can be factored as
  %\begin{equation*}
  %  \cd{
  %  {S} &  
  %}
  %\end{equation*}
  \smallskip

  
  Conversely, assume \(g \colon S \rightarrow T\) is a function
  preserving the operations in \(\Pol(\mathscr{R})\). We need to
  show that \(\gr(g)\) is in \(\mathscr{R}\). Note that we have a span in
  \(\cat{Set}_{f}\) whose right leg is \(g\) and whose
  left leg is \(1_{S}\), and from the universal property of the
  product, a unique function \(h \colon S \rightarrow S \times T\)
  which commutes with the projections. Now, the projections preserve
  the operations in \(\Pol(\mathscr{R})\), as do  \(\pi_{1}h = 1_{S}\)
  and \(\pi_{2}h = g\) by assumption. As we have algebra structure (from
  \(\Pol(\mathscr{R})\)) on \(S \times
  T\) which is pointwise, \(h\) must preserve the
  operations in \(\Pol(\mathscr{R})\). The image of \(h\) is in
  \(\mathscr{R}\) by Lemma~\ref{lem:image-is-rel}, and is equal to the
  graph of \(g\). We conclude that \((k+n,\gr(g))\) is in \(\cat{S}\) as required.
\end{proof}

\subsection{\(\Pol\)-\(\Inv\) and categories of models}
\label{sec:syntactic-categories}

Recall we have a finite product theory \(\cat{T}\) defined from the clone of operations
\(\mathscr{T} =
\Pol(\mathscr{R})\), and a model \(\mathcal{M}\) endowing \(V\) with
its canonical \(\cat{T}\)-algebra structure. We now characterise \(\cat{S}\) as equivalent to a certain
subcategory of the 
category of finite
product preserving functors
\(\FP(\cat{T},\Set)\), that is, the category of \(\cat{T}\)-models.

\begin{Prop}
  \label{prop:forget-S-factorisation}
  Let \(\mathrm{G}\) be the forgetful functor from
  the category of \(\cat{T}\)-models to \(\cat{Set}\) and \(\mathrm{U}
  \) the forgetful functor \(\cat{S}\) to \(\Set\). There is a full and faithful functor \(\mathrm{I}\) making the following diagram of functors
  commute:
  \begin{equation}
    \label{eq:incl-S}
    \cd{
      {\cat{S}} \ar[d]_-{\mathrm{I}
      } \ar[dr]^-{\mathrm{U}
      } & \\
      {\FP(\cat{T},\Set)} \ar[r]_-{\mathrm{G}
      } & {\Set}
    }\rlap{ .}
  \end{equation}
\end{Prop}

\begin{proof}
  It suffices to prove that given \(S \subseteq V^{k}
  \), \(T \subseteq V^{n}
  \) and a function \(h \colon S \rightarrow
  T\) we have:
  \begin{itemize}
  \item \(S
    \) is the underlying set of an object \((k,S)\) in \(\cat{S}\)
    implies it is the underlying set of a subalgebra of
    \(\mathcal{M}^{k}\) (and likewise for \((n,T)\));
  \item Given the above holds, the function \(h\) is the underlying function of a
    morphism in \(\cat{S}\) if and only if it is homomorphism of
    \(\cat{T}\)-algebras. 
  \end{itemize}
  The underlying set of \(\mathcal{M}^{k}\) is \(V^{k}\) for all
  \(k\in \mathbb{N}
  \), thus the first point is equivalent to the characterisation of relations in
  \(\mathscr{R}\) as subalgebras of finite powers of \(V\) under the
  pointwise algebra structure given by \(\mathscr{T}
  \) --- this is
  Corollary~\ref{cor:compatibility-clone-coclone}. The second point is
  Lemma~\ref{lem:mor-iff-preserves-ops}. 
\end{proof}


\begin{Lemma}
  \label{lem:S-subcat-T-alg}
  An object lies in the essential
  image of \(\mathrm{I}
  \) if and only if it admits a monomorphism into a finite power
  of~\(\mathcal{M}\). 
\end{Lemma}

\begin{proof}
  The image factorisation of a monomorphism into a finite power of
  \(\mathcal{M}\) gives an isomorphism with an object in the image of \(\mathrm{I}
  \).
\end{proof}

Taking our inspiration from this result, we define the following:
\begin{Defn}
  \label{def:coclone-of-obj}
  Let \(\mathcal{M}\) be an object in a category \(\cat{C}\) with finite
  products. We define \(\cat{Coclo(\mathcal{M})}\) as the full
  subcategory of \(\cat{C}\) whose objects consist of subobjects of finite
  powers of \(\mathcal{M}\).
\end{Defn}

The equivalence \(\cat{S}\simeq \cat{Coclo}(\mathcal{M})\) makes
proving the next lemma more straightforward than using the definition
of \(\cat{S}\) alone.
\begin{Lemma}
  \label{lem:syntactic-cc-limits}
  We have that \(\cat{S}\simeq \cat{Coclo}(\mathcal{M})\) has finite limits.
\end{Lemma}

\begin{proof}
  The algebra
  \(\mathcal{M}^{0}\) is the terminal object in \(\FP(\cat{T},\Set)\),
  and is in the essential image of \(\mathrm{I}
  \). If
  \(\iota_{A} \colon A \hookrightarrow \mathcal{M}^{k}\) and \(\iota_{B} \colon B
  \hookrightarrow \mathcal{M}^{n}\) are inclusions of subalgebras in
  \(\FP(\cat{T},\Set)\), then we have an inclusion \(\iota_{A}\times \iota_{B} \colon
  A\times B \hookrightarrow \mathcal{M}^{k}\times \mathcal{M}^{n}\) in \(\cat{Coclo}(\mathcal{M})\)
  (This corresponds to the fact that \(\mathscr{R}\) is closed under
  concatenation). Since \(\cat{Coclo}(\mathcal{M})\) is a full
  subcategory, this means it has finite products.
  Also, any equaliser of a parallel pair \(A\rightrightarrows B\) with
  \(\iota_{A}\) and \(\iota_{B}\) (as given above) will admit a monomorphism into
  \(\mathcal{M}^{k}\) by composing with \(\iota_{A}\).
\end{proof}

From Gabriel-Ulmer duality, we obtain a characterisation of the
category of models of \(\cat{S}\):
\begin{Cor}
  \label{cor:coclone-models-lfp}
  Given the syntactic category \(\cat{S}\) defined from a relational
  clone \(\mathscr{R}\), the restricted presheaf category \(\FL(\cat{S},\Set)\) of finite
  limit preserving functors is locally finitely presentable.
\end{Cor}

We now consider how to recover the finite product theory \(\cat{T}\). Note that for each \(n\in \mathbb{N}
\), the free algebra \(\cat{T}([n,V],\thg)\) is a subalgebra of
\(\mathcal{M}^{V^{n}}\) as \(\mathscr{T}(n)
\) is a subset of \(V^{V^{n}}\) compatible with each operation in \(\mathscr{T}
\). Indeed, the free algebras on
finitely many generators form a full subcategory of \(\cat{Coclo}(\mathcal{M})\). Thus there is an
object \(\mathcal{W}\) in \(\FL(\cat{S}, \cat{Set})\) corresponding
to the free algebra on one generator in
\(\cat{Coclo}(\mathcal{M})\), and \([1,V]\) in \(\cat{T}\). It is now
natural to make the following definition.

\begin{Defn}
  \label{def:clone-of-obj}
  Let \(\mathcal{W}\) be an object in some category \(\cat{C}\) with finite
  products. We define \(\cat{Clo(\mathcal{W})}\) as the full subcategory of
  \(\cat{C}\) whose objects are finite powers of \(\mathcal{W}\).
\end{Defn}

\begin{Prop}
  \label{prop:clo-coclo-equiv}
  Given a complementary clone and coclone \(\mathscr{T} = \Pol(\mathscr{R}
  )
  \) and \(\mathscr{R} = \Inv(\mathscr{T}
  )\) on a finite set \(V\), we have a finite product
  theory \(\cat{T}\) and a finite limit theory \(\cat{S}\); Moreover, there is
  an object \(\mathcal{W}\) in \(\FL(\cat{S},\Set)\) and an object
  \(\mathcal{M}\) in \(\FP(\cat{T}, \cat{Set})\) such
  that we have equivalences \(\cat{Clo(\mathcal{W})} \simeq \cat{T}\) and
  \(\cat{Coclo(\mathcal{M})} \simeq \cat{S}\).
\end{Prop}

\subsection{Categorical formulation}
\label{sec:categ-form}

We now consider our setting entirely in categorical terms.
\begin{Defn}[Locally finite category]
  \label{def:lfc}
  A category is called \emph{locally
    finite} if it has finite homsets.
\end{Defn}
%.
This terminology is entirely consistent with the definition of locally
  finite varieties in universal algebra, which are varieties in which
  finitely generated free algebras are finite. Indeed, locally finite
  varieties are precisely those which are models of a locally finite
  (finite product) theory. We therefore have:
  %.
\begin{Prop}
  \label{prop:concr-clo-loc-fin-theory}
  A finite product theory is equivalent to the syntactic category of a
  concrete clone on a finite set if and only if it is a locally finite category. 
\end{Prop}

\begin{Prop}
  \label{prop:ct-concr-clo}
  Let \(\cat{T}\) be a finite product (Lawvere) theory which is a
  locally finite category. Assume we have a
faithful finite product preserving functor \(\mathcal{M} \colon
\cat{T} \rightarrow \Set\).
Then \(\cat{T}\) is equivalent to \(\cat{Clo}(\mathsf{W})\) for some
\(\mathsf{W}\) in the category
of models of \(\cat{Coclo}(\mathcal{M})\). Moreover, we have 
\(\FL(\cat{Coclo}(\mathcal{M}),\Set)( \mathrm{Y}
  \mathcal{M}, \iota(\thg)) \cong \mathcal{M}\) where \(\iota\)
  denotes the inclusion and \(\mathrm{Y}
  \) is the Yoneda embedding functor.
\end{Prop}

To say we have a concrete clone on a finite set \(V\) is to say we
have a finite
product (Lawvere) theory \(\cat{T}\) which is a locally finite category, and a
faithful finite product preserving functor \(\mathcal{M} \colon \cat{T} \rightarrow \Set\).

Take
the full subcategory
%\(\cat{K} \defeq \cat{Coclo}(\mathcal{M})\)
\( \cat{Coclo}(\mathcal{M})\)
of \(\FP(\cat{T},
\cat{Set})\); it has finite limits. For any \(A \in
\FP(\cat{T},\Set)\), we can precompose
%\(\mathcal{W}\) such that
the hom-functor with the inclusion, as in the diagram:
\begin{equation}
  \label{eq:coclo-model}
  \xymatrixcolsep{6pc}
  \xymatrixrowsep{3.5pc}
  \cd{
    %{\cat{T}^{\mathrm{op}}} \ar[r]^-{} \ar[dr]_-{\mathrm{Y}
    %} & 
    {\cat{Coclo}(\mathcal{M})} \ar[d]_-{\mathrm{I}
    } \ar[dr]
    %^-{\mathcal{W}}
    & \\
    %&
    {\FP(\cat{T},\Set)} \ar[r]_-{\FP(\cat{T},\Set)(
      %\mathrm{Y}\aone
      A,\thg)
    } & {\Set}
    \rlap{ .}
  }
\end{equation}
This functor is a model of \(\cat{Coclo}(\mathcal{M})\); as a full and faithful functor, \(\mathrm{I}
\) preserves limits, as do covariant
hom-functors, so their composition certainly preserves finite limits.
Let \(\mathrm{Y}
\colon \cat{T} \rightarrow \FP(\cat{T},\Set)\) be the Yoneda
embedding, with \(\mathrm{Y}\aone\) thus denoting the free algebra on
one generator (following the usual convention with finite product
theories). Take diagram~\eqref{eq:coclo-model} with \(A =
\mathsf{Y}\aone\).
%This is simply the
%forgetful functor from \(\FP(\cat{T},\Set)\) to \(\cat{Set}\), so for any monomorphism \(\mathcal{A}\rightarrow \mathcal{M}^{k}\), we have a monomorphism
%\(\FP(\cat{T},\Set)(\mathrm{Y}
%\aone,\mathcal{A}) \rightarrow V^{k}\).

We now write \(\mathsf{W} \defeq \FP(\cat{T},\Set)(\mathrm{Y}
\aone,\mathrm{I}
(\thg))\).
%and observe that \(\mathsf{W}\) is an object in the category
%\(\FL(\cat{Coclo}(\mathcal{M}),\Set)\). 
Take \(\cat{Clo}(\mathsf{W})\) in
\(\FL(\cat{Coclo}(\mathcal{M}),\Set)\).  In particular, limit
preserving functors, preserve finite products, so for any \(\mathsf{X} \in
\FL(\cat{Coclo}(\mathcal{M}),\Set)\), we can precompose
%\(\mathcal{W}\) such that
the hom-functor with the inclusion and obtain a model of the finite
product theory \(\cat{Clo}(\mathsf{W})\):
\begin{equation}
  \label{eq:clo-model}
  \xymatrixcolsep{8pc}
  \xymatrixrowsep{3.5pc}
  \cd{
    {\cat{Clo}(\mathsf{W})}
    \ar[d]_-{\iota} \ar[dr]^-{}
    %{\cat{K}(\thg,\mathcal{M})}
    & \\
    {\FL(\cat{Coclo}(\mathcal{M}),\Set)} \ar[r]_-{\FL(\cat{Coclo}(\mathcal{M}),\Set)( \mathsf{X}, \thg)
    } & {\Set}
  }\rlap{ .}
\end{equation}
%we have a model of the theory \(\cat{Clo}(\mathsf{W})\),
Take \(\mathsf{X}\) in diagram~\eqref{eq:clo-model} to be \(\mathrm{Y}(\mathcal{M})
\), where we use \(\mathrm{Y}
\) again to denote the Yoneda embedding. We have
\(\cat{Clo}(\mathsf{W}) \simeq \cat{T}\).
Moreover,
\begin{equation}
  \label{eq:M-equation}
  \FL(\cat{Coclo}(\mathcal{M}),\Set)( \mathrm{Y}
  \mathcal{M}, \iota(\thg)) \cong \mathcal{M}
\end{equation}
and
\begin{equation}
  \label{eq:W-equation}
    \FP(\cat{Clo}(\mathsf{W}),\Set)(\mathrm{Y}
  \mathsf{W},\mathrm{I}
  (\thg)) \cong \mathsf{W} \rlap{ .}
\end{equation}
%\(\FL(\cat{Coclo}(\mathsf{W}),\Set)( \mathrm{Y}
%\mathcal{M}, \iota(\thg)) \cong \mathcal{M}\) and \(\FP(\cat{Clo}(\mathsf{W}),\Set)(\mathrm{Y}
%\mathsf{W},\mathrm{I}
%(\thg)) = \mathsf{W} \).
Let \(N\) be equal to the
expression on the left-hand side of~\eqref{eq:M-equation}. It is
straightforward that
\(\cat{Coclo}(N) \simeq \cat{Coclo}(\mathcal{M})\). Therefore we can
go from `categorical concrete clone' to `categorical coclone' back to
`categorical concrete clone', and get the same thing up to
isomorphism. It is less clear with what we have so far how to go from `categorical coclone' back to
`categorical concrete clone' then back again and get essentially the
same thing. How we can have this work will be made clear in
Section~\ref{sec:duality}. We will first find a presentation of the
finitely presentable objects in the category of models of
\(\cat{Coclo}(\mathcal{M})\). This involves looking at the category of
structures over the relational signature obtained from the relational clone \(\mathscr{R}\).

\section{Presentations via signatures}
\label{sec:via-sig-structures}


\subsection{Setting: \(\Omega\)-algebras and \(\Psi\)-structures}
\label{sec:sig-setting}


We will continue to consider a clone of operations \(\mathscr{T}
\subseteq \O_{V}\) and a clone of
relations \(\mathscr{R}
\subseteq \R_{V}\) on a finite set \(V\) such that  \(\Pol(\mathscr{R}
) = \mathscr{T}
\) and \(\Inv(\mathscr{T}
) = \mathscr{R}
\). From the previous section, we also have the finite product theory
\(\cat{T}\) constructed from \(\mathscr{T}
\), and finite limit theory \(\cat{S}\) constructed from \(\mathscr{R}\).

In addition, we have a (purely operational) signature \(\Omega\) consisting of
operation symbols and arities assigned from \(\mathscr{T}
\). Similarly, we have a (purely relational) signature \(\Psi\) consisting of
relation symbols and arities from \(\mathscr{R}
\). 

\begin{Defn}[The category \(\cat{\Omega St}\)]
  \label{def:Om-str}
  We define $\cat{\Omega St}$ as the category whose objects \(\mathsf{A}\) are
  sets \(A\) endowed with an operation
  \(\dbr{\tau} \colon A^{n} \rightarrow A\) for each \(\tau\in \mathscr{T}(n)\), and whose
  morphisms are maps commuting with the interpretations. We write
  \(\cat{\Omega St}_{f}\) for the category of finite
  \(\Omega\)-structures (\(\Omega\)-algebras). Note that \(\cat{\Omega St}_{f}\)
  has finite products, where the interpretations of operations are
  given pointwise.
\end{Defn}

\begin{Defn}[The category \(\cat{\Psi St}\)]
  \label{def:psi-str}
  We define \(\cat{\Psi St}\) as the category whose objects \(\mathsf{X}\) are
  sets \(X\) endowed with a relation \(\dbr{S} \subseteq X^{k}\) for each
  \(S\in \mathscr{R}(k)
  \), and whose morphisms are maps which preserve the
  relations. Note that we write \(\cat{\Psi St}_{f}\) for the category
  of \(\Psi\)-structures with finite underlying sets, and that this
  category has finite products where the
  interpretation of relations is given (pointwise) by the cartesian product.
\end{Defn}


\begin{Defn}[\(\mathscr{R}
  \) inside $\cat{\Omega St}$]
  \label{def:rel-in-F-str}
  We write $\mathsf{M}$ for the \(\Omega\)-structure \(\langle V;
  \Omega \rangle\), that is, the object with underlying set $V$
  and interpretation $\dbr{\tau} = \tau$. For any relation $S \in \mathscr{R}(k)$, its
  compatibility with each operation in \(\mathscr{T}
  \) means that we have a
  \(\Omega\)-subalgebra
  \begin{equation*}
    \cd{
      {\mathsf{S}\,} \ar@{^{(}->}[r]^-{\iota_{S}}
      %\ar@{ >->}[r]^
      & \mathsf{M}^{k} \rlap{ .}
    }
  \end{equation*}
\end{Defn}



\begin{Defn}[\(\mathscr{T}
  \) inside $\cat{\Psi St}$]
  \label{def:ops-in-R-str}
  We write $\mathsf{W}$ for the \(\Psi\)-structure \(\langle V; \Psi
  \rangle\), that is, the object with underlying set $V$
  and interpretations $\dbr{S} = S$. For any \(n\)-ary operation
  \(\tau \in \mathscr{T}
  \), its compatibility with each relation in \(\mathscr{R}\) means that we
  have a
  \(\Psi\)-structure morphism
  \begin{equation*}
    \cd{
      \mathsf{W}^{n} \ar[r]^-{\tau} & \mathsf{W}
      \rlap{ .}}
  \end{equation*}
\end{Defn}

It follows from Definition~\ref{def:rel-in-F-str} that $\mathsf{M}$ is an
internal 
\(\Psi\)-structure in $\cat{\Omega St}$, and it follows from Definition~\ref{def:ops-in-R-str}  that $\mathsf{W}$
is an internal \(\Omega\)-structure (\(\Omega\)-algebra) in $\cat{\Psi St}$. Thus, we can lift the hom-functors
\begin{equation*}
  \cat{\Omega St}_{f}(\thg, \mathsf{M}) \colon \cat{\Omega St}_{f}^\mathrm{op} \rightarrow \cat{Set}_{f} \qquad \cat{\Psi St}_{f}(\thg, \mathsf{W}) \colon \cat{\Psi St}_{f} \rightarrow \cat{Set}_{f}^\mathrm{op}
\end{equation*}
to functors
\begin{equation*}
  \mathsf{M}^{(\thg)} \colon \cat{\Omega St}_{f}^\mathrm{op} \rightarrow
  \cat{\Psi St}_{f} \qquad 
  \mathsf{W}^{(\thg)} \colon \cat{\Psi St}_{f} \rightarrow
  \cat{\Omega St}_{f}^\mathrm{op}\rlap{ .}
\end{equation*}

\begin{Lemma}
  \label{lem:adj-M-W-lifted-homs}
  We have $\mathsf{W}^{(\thg)} \dashv \mathsf{M}^{(\thg)}$.
\end{Lemma}

\begin{proof}
  For an \(\Omega\)-algebra $\mathsf A$ and \(\Psi\)-structure $\mathsf X$, to give a map of \(\Omega\)-algebras
  \begin{equation*}
    \mathsf{A} \rightarrow \mathsf W^{\mathsf{X}}
  \end{equation*}
  is to give a function \(\theta \colon A \times X \rightarrow V\) such
  that:
  \begin{itemize}
  \item For any \(a\in A\), the function $\theta(a, \thg) \colon X
    \rightarrow V$ is a \(\Psi\)-structure map $\mathsf{X} \rightarrow \mathsf{W}$;
  \item For any \(x\in X\), the function $\theta(\thg, x) \colon A
    \rightarrow V$ is an \(\Omega\)-algebra map $\mathsf{A} \rightarrow \mathsf{M}$.
  \end{itemize}
  Symmetrically, to give a map of \(\Psi\)-structures
  \begin{equation*}
    \mathsf{X} \rightarrow \mathsf{M}^\mathsf{A}
  \end{equation*}
  is to give a function \(\theta \colon A \times X \rightarrow V\) such
  that:
  \begin{itemize}
  \item For any \(a\in A\), the function $\theta(a, \thg) \colon X
    \rightarrow V$ is a \(\Psi\)-structure map $\mathsf{X} \rightarrow \mathsf{W}$;
  \item For any \(x\in X\), the function $\theta(\thg, x) \colon A
    \rightarrow V$ is an \(\Omega\)-algebra map $\mathsf{A} \rightarrow \mathsf{M}$.
  \end{itemize}
  These correspondences are easily seen to be natural in $\mathsf{A}$ and $\mathsf{X}$.
\end{proof}

\begin{Lemma}
  \label{lem:WexpY-in-ISPfM}
  For any \(\mathsf{X} \in \cat{\Psi St}_{f}\), the object \(\mathsf{W}^{\mathsf{X}} \in
  \cat{\Omega St}\) lands in \(\cat{Coclo}(\mathsf{M})\).
\end{Lemma}
\begin{proof}
  
  Recall we have free and forgetful functors
  \begin{equation}
    \label{eq:psi-free-forget}
    %\xymatrixcolsep{4pc}
    %\xymatrixrowsep{3pc}
    \cd{
      {\cat{\Psi St}}
      \ar@<-4.5pt>[r]_-{\mathrm{G}^{\Psi}}
      \ar@{<-}@<4.5pt>[r]^{\mathrm{F}^{\Psi}} \ar@{}[r]|-{\bot} &
      {\Set}
      \rlap{ .}
    }
  \end{equation}
  Let \(\mathsf{X}\in \cat{\Psi St}_{f}\), and consider the free
  \(\Psi\)-structure
  %\(\mathrm{F}^{\Psi}\mathrm{G}^{\Psi}\mathrm{X}\)
  \(\mathrm{F}^{\Psi}X\) on the underlying (finite) set of
  \(\mathsf{X}\). Note that it has the same underlying set as
  \(\mathsf{X}\), but
  all relations are empty. The counit  \(\mathrm{F}^{\Psi}\mathrm{G}^{\Psi}
  \mathsf{X}
  \rightarrow \mathsf{X}\) is an epimorphism
  %The identity function {\blue [not sure I
  %understand this]} is an epimorphism
  %$R (Y) \rightarrow \mathsf{Y}$,
  whose image under
  $\mathsf{W}^{(\thg)}$ is a monomorphism
  $\mathsf{W}^{\mathsf{X}} \hookrightarrow \mathsf{M}^{\mathrm{G}^{\Psi}(\mathsf{X})}$, as desired.
\end{proof}


\begin{Prop}
  \label{prop:adj-M-W-counit-invble}
  The counit of the adjunction
  $\mathsf{W}^{(\thg)} \dashv \mathsf{M}^{(\thg)}$ is invertible at
  any \(\mathsf{A} \in \cat{Coclo}(\mathsf{M})\).
\end{Prop}

\begin{proof}
  Let \(\mathsf{A}\in \cat{\Omega St}_{f}\). As \(\Omega\) consists
  only of operations, it suffices to show that the counit
  $\varepsilon_{\mathsf{A}} \colon \mathsf{A} \rightarrow
  \mathsf{W}^{\mathsf{M}^{\mathsf{A}}}$ is an invertible function. The object $\mathsf{W}^{\mathsf{M}^{\mathsf{A}}} \in \cat{\Omega St}$ is given
  by the set of functions $\theta \colon \cat{\Omega St}(\mathsf{A}, \mathsf{M}) \rightarrow V$
  such that, for all $S \in \mathscr{R}
  (k)$ and $f_1, \dots, f_k \in
  \cat{\Omega St}(\mathsf{A}, \mathsf{M})$, we have
  \begin{equation}
    \label{eq:compat:prop:counit-inv}
    (f_1(a), \dots, f_k(a)) \in S \text{ for all }a \in A \quad \implies \quad
    (\theta(f_1), \dots, \theta(f_k)) \in S\rlap{ .}
  \end{equation}
  In particular, for all \(a\in A\), the evaluation function
  \begin{equation*}
    \mathsf{ev}_{a} \colon f \in
    \cat{\Omega St}
    (\mathsf{A}, \mathsf{M}) \quad \mapsto \quad f(a) \in V
  \end{equation*}
  is in $\mathsf{W}^{\mathsf{M}^{\mathsf{A}}}$, and the counit
  \(\varepsilon_{\mathsf{A}}\) 
  sends \(a\) to $\mathsf{ev}_{a}$. 

  To see $\varepsilon_{\mathsf{A}}$ is injective, let us begin by
  expressing $\mathsf{A}$ as a subobject 
  \begin{equation*}
    \cd{\mathsf{A} \ar@{ >->}[r]^-{\iota} & \mathsf{M}^n}
  \end{equation*}
  for some \(n\in \mathbb{N}
  \), and write
  $\iota_1, \dots, \iota_n \colon \mathsf{A} \rightarrow \mathsf{M}$
  for the composites of $\iota$ with the $n$ projection maps. Now, if
  $\mathsf{ev}_{a} = \mathsf{ev}_{b} \in
  \mathsf{W}^{\mathsf{M}^{\mathsf{A}}}$, then
  $\iota_k(a) = \mathsf{ev}_{a}(\iota_k) = \mathsf{ev}_{b}(\iota_k) =
  \iota_k(b)$ for all $k\in [n]$, and so, since $\iota$ is injective,
  $a = b$.

  This proves $\varepsilon_{\mathsf{A}}$ is injective; we now show surjectivity.
  Without loss of generality, we may assume $\iota$ is a subset
  inclusion; whence, since $\mathscr{R}
  = \mathsf{Inv}(\mathscr{T}
  )$, we have
  $A \subseteq V^n$ is a relation in $\mathscr{R}
  (n)$. Taking \(S\) in~\eqref{eq:compat:prop:counit-inv}
  to be \(A\), and the $f_i$'s to be $\iota_1, \dots, \iota_n$,
  it is obvious that $(\iota_1(a), \dots, \iota_n(a)) \in A$ for all
  \(a\in A\), whence $a_\theta \defeq (\theta(\iota_1), \dots, \theta(\iota_n)) \in A$
  also.

  We now show that $\theta = \mathsf{ev}_{a_\theta}$. To this end, let
  $f \colon \mathsf{A} \rightarrow \mathsf{M}$ be a homomorphism of
  \(\Omega\)-structures.
  %, and consider 
  %\begin{equation*}
  %  \mathrm{gr}(f) = \{(\vec v, w) \in V^{n+1} : \vec v \in A,\, f(\vec
  %  v) = w\} \subseteq V^{n+1}
  %\end{equation*}
  We have $\mathrm{gr}(f) \in \mathscr{R}(n+1)$ by
  Lemma~\ref{lem:mor-iff-preserves-ops}, and evidently
  \begin{equation*}
    \bigl(\iota_1(a), \dots, \iota_n(a), f(a)\bigr) \in \mathrm{gr}(f) \quad \text{for all \(a\in A\).}
  \end{equation*}
  Thus by~\eqref{eq:compat:prop:counit-inv} we have
  \begin{equation*}
    (\theta(\iota_1), \dots, \theta(\iota_n), \theta(f)) \in \mathrm{gr}(f)
  \end{equation*}
  which is to say that $f(x_\theta) = \theta(f)$, and so $\theta =
  \mathrm{ev}_{x_\theta}$ as required.
\end{proof}

From Lemma~\ref{lem:WexpY-in-ISPfM} and Proposition~\ref{prop:adj-M-W-counit-invble}, it follows that
each counit map 
\begin{equation*}
  \mathsf{W}^\mathsf{X} \rightarrow
  \mathsf{W}^{\mathsf{M}^{\mathsf{W}^{\mathsf{X}}}}
\end{equation*}
is invertible; so
the adjunction $\mathsf{W}^{(\thg)} \dashv \mathsf{M}^{(\thg)}$ is
\emph{idempotent}. In particular, this means that the unit of the
adjunction is invertible at precisely those objects in the essential
image of $\mathsf{M}^{(\thg)}$. We now characterise those.
\begin{Lemma}
  \label{lem:ob-in-im-Mexp-eq}
  An object $\mathsf{X} \in \cat{\Psi St}_{f}$ is in the essential image of
  $\mathsf{M}^{(\thg)}$ if and only if it can be expressed as an equaliser
  of maps between finite powers of $\mathsf{W}$:
  \begin{equation}
    \label{eq:eq-in-R-str}
    \cd{
      \mathsf{X} \ar@{ >->}[r]^-{\iota} & \mathsf{W}^n \ar@<3pt>[r]^-{f} \ar@<-3pt>[r]_-{g} & \mathsf{W}^m
    }
  \end{equation}
\end{Lemma}

\begin{proof}
  Suppose first that $\mathsf{X} \cong \mathsf{M}^{\mathsf{A}}$ for
  some $\mathsf{A} \in \cat{\Omega St}_{f}$.
  Consider the free \(\Omega\)-structure
  $FU(\mathsf{A})  \in \cat{\Omega St}_{f}$ on the underlying set of
  \(\mathsf{A}\).
   We have a coequaliser diagram of
  the form
  \begin{equation*}
    \cd{
      FUFU(\mathsf{A}) \ar@<3pt>[r]^-{} \ar@<-3pt>[r]_-{} & FU(\mathsf{A}) \ar@{->>}[r]^-{} & \mathsf{A}
      \rlap{ .}
    }
  \end{equation*}
  Note that all of these
  \(\Omega\)-structures are indeed finite. 
  Applying the functor $\mathsf{M}^{(\thg)}$, we get an equaliser
  diagram in $\cat{\Psi St}_{f}$ of the form~\eqref{eq:eq-in-R-str},
  where $n = U(\mathsf{A})$ and
  \(m = UFU(\mathsf{A})\).
  %{\blue check? \(m = UFn\)?}. 

  Conversely, assume that $\mathsf{X} \in \cat{\Psi St}$ has an expression as an
  equaliser~\eqref{eq:eq-in-R-str}. Since $\mathscr{T}
  = \mathsf{Pol}(\mathscr{R}
  )$, the
  components $f_1, \dots, f_m, g_1, \dots, g_m \colon \mathsf{W}^n \rightarrow
  \mathsf{W}$ of $f$ and $g$ are operations in $\mathscr{T}
  (n)$. Thus we have a
  parallel pair of maps in $\cat{\Omega St}_{f}$
  \begin{equation*}
    \cd{
      F(m) \ar@<3pt>[r]^-{\varphi} \ar@<-3pt>[r]_-{\gamma} & F(n)\rlap{ ,}
    }
  \end{equation*}
  where $\varphi(\pi_i) = f_i$ and $\gamma(\pi_i) = g_i$, whose image
  under $\mathsf{M}^{(\thg)}$ is the parallel pair $(f,g)$. Thus, if
  we let $\mathsf{Q} \in \cat{\Omega St}$ be the coequaliser of $\varphi$ and
  $\gamma$, then its image under $\mathsf{M}^{(\thg)}$ will be
  isomorphic to $\mathsf{X}$, as claimed.
\end{proof}

\begin{Cor}
  \label{cor:adj-M-W-restr-equiv}
  The adjunction $\mathsf{W}^{(\thg)} \dashv \mathsf{M}^{(\thg)}$
  restricts to an equivalence of categories between
  \(\cat{Coclo}(\mathsf{M})^{\mathrm{op}}\) and the full subcategory of \(\cat{\Psi St}_{f}\) on
  those $\mathsf{X}$ which can be expressed as an equaliser
  diagram~\eqref{eq:eq-in-R-str}.
\end{Cor}

It is easily observed that  \(\cat{Coclo(\mathsf{M})} \simeq
\cat{Coclo(\mathcal{M})} \), so we also have that:
\begin{Cor}
  \label{cor:dual-S-W}
  The full subcategory of \(\cat{\Psi St}_{f}\) on
  those $\mathsf{X}$ which can be expressed as an equaliser
  diagram~\eqref{eq:eq-in-R-str} is dually equivalent to
  \(\cat{Coclo}(\mathcal{M}) \simeq \cat{S}\).
\end{Cor}



\begin{Lemma}
  \label{lem:33}
  The full subcategory of \(\cat{\Psi St}_{f}\) on
  those $\mathsf{X}$ which can be expressed as an equaliser
  diagram~\eqref{eq:eq-in-R-str} is a reflective subcategory.
\end{Lemma}

\begin{proof}
  From Corollary~\ref{cor:adj-M-W-restr-equiv}, we know that the
  subcategory described
  is equivalent to \(\cat{Coclo}(\mathsf{M})\),
  and so its inclusion into \(\cat{\Psi St}_{f}\) has a left adjoint. In
  particular, the reflector is given by the unit $\mathsf{X}
  \rightarrow \mathsf{M}^{\mathsf{W}^{\mathsf{X}}}$ of the adjunction
  $\mathsf{W}^{(\thg)} \dashv \mathsf{M}^{(\thg)}$.
\end{proof}


Recall we have syntactic category \(\cat{S}\) defined from the
relational clone \(\mathscr{R}\). An object \(Y\) of
\(\FL(\cat{S},\Set)\) admits a presentation as
a structure for the signature \(\Psi\), with the interpretation of a
relation \(S \in \mathscr{R}(k)
\) as a relation
\begin{equation*}
  \dbr{S} =
  \FL(\cat{S},\Set)(\mathrm{Y}
  (k,S),Y)
\end{equation*}
%\(\dbr{S} =
%\FL(\cat{S}^{\mathrm{op}},\Set)(\yo(k,S),Y)\)
where \(\mathrm{Y}
\) denotes
the restricted Yoneda embedding. Naturally, the morphisms in
\(\FL(\cat{S},\Set)\) preserve the relations in
\(\Psi\), and so we obtain a forgetful functor \(\FL(\cat{S},\Set)
\rightarrow \cat{\Psi St}\).


\section{Interlude}
\label{sec:interlude}

We continue to assume we are given a clone of operations \(\mathscr{T}
\subseteq \O_{V}\) and a clone of
relations \(\mathscr{R}
\subseteq \R_{V}\) on a finite set \(V\), with  \(\Pol(\mathscr{R}
) = \mathscr{T}
\) and \(\Inv(\mathscr{T}
) = \mathscr{R}
\). We also have the finite product theory
\(\cat{T}\) constructed from \(\mathscr{T}
\), and finite limit theory \(\cat{S}\) constructed from
\(\mathscr{R}\) in Section~\ref{sec:syntactic-categories}. In the
same section, we introduced the canonical model \(\mathcal{M}\) in
\(\FP(\cat{T},\Set)\) endowing \(V\) with \(\cat{T}\)-algebra
structure, and \(\mathcal{W}\) in \(\FL(\cat{S}, \cat{Set})\) endowing \(V\) with \(\cat{S}\)-model
structure.


\subsection{The story so far}
\label{sec:story-so-far}

\begin{enumerate}[label=(\alph*), ref=(\alph*)]
\item
  \label{it:pol-inv-bij}
  The fixed points on each side of the \(\Pol\)-\(\Inv\) Galois
  connection are clones of operations \(\langle V; \Omega 
  \rangle \) and clones of relations \(\langle V; \Psi \rangle\),
  and these are in bijection when the base set \(V\) is finite;
  
\item
  \label{it:pol-inv-syntactic-cats}
  We can construct syntactic categories \(\cat{T}\) from \(\mathscr{T}
  \) and \(\cat{S}\) from \(\mathscr{R}\), with equivalences \(\cat{T} \simeq \cat{Clo}
  (\mathcal{W})\) and \(\cat{S} \simeq \cat{Coclo}(\mathcal{M})\)
  where concrete clone \(\mathscr{T} = \Pol(\mathscr{R}
  )
  \) and
  relational clone \(\mathscr{R}= \Inv(\mathscr{T}
  )\) on \(V\), and these are equipped with faithful functors into
  \(\cat{Set}_{f}\) endowing \(V\) with \(\cat{T}\)-model structure
  corresponding to model \(\mathcal{M}\) and
  \(\cat{S}\)-model structure corresponding to model \(\mathcal{W}\);
  
\item
  \label{it:pol-inv-syntax-4}
  Given any of the four listed above, we can recover the other
  three;
  
\item
  \label{it:pol-inv-sig-method}
  We can construct \(\cat{T} \simeq \cat{Clo}
  (\mathcal{W})\) and \(\cat{S} \simeq \cat{Coclo}(\mathcal{M})\) up to equivalence as
  subcategories of 
  the categories of \(\Omega\)-structures and
  \(\Psi\)-structures respectively and their models as subcategories of 
  the categories of \(\Psi\)-structures and
  \(\Omega\)-structures, respectively;
  
\item
  \label{it:models-lfp}
  The coclone theory (syntactic category) \(\cat{S}\)
  %and \(\cat{T}\) admit nice
  has finite limits and
  so has a well developed theory of functorial semantics under
  Gabriel-Ulmer duality --- its models form a
  locally finitely presentable category;

\item
  \label{it:T-not-enough}
  Although we always have a full and faithful functor
  \(\cat{T}^{\mathrm{op}} \rightarrow \cat{S}\), the category
  \(\cat{T}\) is \emph{not}, in general, enough to reconstruct
  \(\cat{S}\) --- the situation is asymmetric;
\end{enumerate}

This is not altogether satisfying from a categorical perspective. We
don't have an equivalence of categories capturing the duality between
clones and coclones as apparent in the setting of the
\(\Pol\)-\(\Inv\) Galois connection.

It turns out we can rectify this situation; this
will be the topic of Section~\ref{sec:duality}. First, we will need to know more
about \(\cat{Coclo}(\mathcal{M}
)\), and a particular class of finite limit theories and their
categories of models, namely \emph{quasivarieties}. 


\subsection{Quasivarieties, quasivarietal theories, and the dual of a coclone}
\label{sec:quasivariety}

\begin{Defn}[Quasivariety]
  \label{def:quasivariety}
  A (finitary, single-sorted) \emph{quasivariety} is a class of
  algebras of some fixed (purely operational) signature \(\Omega\), defined by
  \emph{quasiequations}, that is, implications of the following form:
  \begin{equation}
    \label{eq:21}
    \bigwedge_{i\in [n]} (x_{i} = y_{i}) \implies x = y
  \end{equation}
  where \(x_{i},y_{i},x,y\) are all \(\Omega\)-terms and \(n\in \mathbb{N}
  \).
\end{Defn}

\begin{Ex}
  \label{ex:variety-is-qv}
  Every equation may be considered a quasiequation in which \(n
  = 0\). Therefore any variety is a quasivariety.
\end{Ex}

\begin{Ex}
  \label{ex:left-semigrp}
  Left-cancellative semigroups are semigroups which satisfy the following
  quasiequation:
  \begin{equation*}
    \label{eq:22}
    (x \cdot y = x \cdot z) \implies y = z.
  \end{equation*}
\end{Ex}

Various characterisations of quasivarieties in categorical terms have
been given, with the first attributed to Isbell~\cite{Isbell:1964aa}. The formulation
we present below is Corollary 4.6 in~\cite{Pedicchio:2000aa}.
\begin{Prop}
  \label{prop:quasivariety-reg-epi-refl}
  For any category \(\cat{A}\), the following are equivalent:
  \begin{enumerate}[label=(\alph*), ref=(\alph*)]
  \item
    \label{it:pedic-qv-gen}
    \(\cat{A}\) is a regular category with a finitely presentable,
    regular projective, regular generator;
  \item
    \label{it:pedic-qv-reg-epi-refl}
    \(\cat{A}\) is equivalent to a regular epi-reflective subcategory of a variety, and the inclusion
    functor preserves filtered colimits;
  \item
    \label{it:pedic-qv}
    \(\cat{A}\) is equivalent to a quasivariety.
  \end{enumerate}
\end{Prop}
As we might expect from Example~\ref{ex:variety-is-qv},
\(\FP(\cat{T},\Set)\) automatically satisfies the conditions to be a
quasivariety, most obviously Item~\ref{it:pedic-qv-reg-epi-refl}.
Let \(\cat{A}_{f}\) denote the full subcategory of \(\FP(\cat{T},
\cat{Set})\) whose objects are algebras with finite underlying sets. As a locally finite variety, the algebras with finite underlying set are precisely the
finitely presentable objects in the category of \(\cat{T}\)-models \(\FP(\cat{T},
\cat{Set})\) (which is indeed locally finitely presentable).  Now, the free and forgetful adjunction may be written as
\begin{equation}
  \label{eq:hom-adj-free-gen}
  \xymatrixcolsep{5pc}
  \xymatrixrowsep{3.5pc}
  \cd{
    {\FL(\cat{A}_{f}^{\mathrm{op}},\cat{Set})}
    \ar@<-4.5pt>[r]_-{\FL(\cat{A}_{f}^{\mathrm{op}},\cat{Set})(\mathcal{G}, \thg)} \ar@{<-}@<4.5pt>[r]^-{\mathcal{G}\ast (\thg)} \ar@{}[r]|-{\bot} &
    {\Set}
    \rlap{ ,}
  }
\end{equation}
where \(\mathcal{G}\) denotes the free algebra on one generator, and
\(\ast\) denotes copower. We have that \(\mathcal{G}\) is finitely
presentable and every object in \(\cat{A}
\defeq \FL(\cat{A}_{f}^{\mathrm{op}},\cat{Set})\). is a regular
quotient of a copower of \(\mathcal{G}\) --- this is the object of Item~\ref{it:pedic-qv-gen}.

The
following characterisation of quasivarietal theories will play an important role.
It is Corollary 2 in~\cite{Adamek:1998aa}.
\begin{Prop}
  \label{prop:qv-flt}
  An essentially small category is equivalent to the theory of a (single-sorted)
  quasivariety if it has finite limits and contains a single object
  \(I\) such that
  \begin{enumerate}[label=(\roman*), ref=(\roman*)]
  \item
    \label{it:qv-reg-inj}
\(I\) is regular injective (and so every power \(I^{n}\) is
    regular injective);
  \item
    \label{it:qv-reg-subobj}
Every object is a regular subobject of \(I^{n}\) for some
    \(n \in \mathbb{N}\).
  \end{enumerate}
\end{Prop}


Note that \(\cat{A}_{f}\) is finitely complete, once again, as it is a
locally finite variety, so we have restricted hom-adjunctions
\begin{equation}
  \label{eq:hom-adj-cogen-gen}
  \xymatrixcolsep{5pc}
  \xymatrixrowsep{3.5pc}
  \cd{
    {\Set_{f}^{\mathrm{op}}}
    \ar@<-4.5pt>[r]_-{\mathcal{M}^{(\thg)}} \ar@{<-}@<4.5pt>[r]^-{\cat{A}_{f}(\thg,\mathcal{M})} \ar@{}[r]|-{\bot} &
    {\cat{A}_{f}}
    \ar@<-4.5pt>[r]_-{\cat{A}_{f}(\mathcal{G}, \thg)} \ar@{<-}@<4.5pt>[r]^-{\mathcal{G}\ast (\thg)} \ar@{}[r]|-{\bot} &
    {\Set_{f}}
    \rlap{ .}
  }
\end{equation}



\begin{Lemma}
  \label{lem:coclo-reg-epi-reflect-subcat}
  Consider the full subcategory of \(\cat{A}_{f}\) whose objects are
  those algebras \(A\) for which the unit \(\eta_{A} \colon A \rightarrow \mathcal{M}^{\cat{A}_{f}(A,\mathcal{M})}\)
  is a monomorphism. This category is a regular epi-reflective
  subcategory of \(\cat{A}_{f}\), and is equivalent to \(\cat{Coclo(\mathcal{M})}\).
\end{Lemma}

\begin{proof}
  Observe that \(\cat{A}_{f}\) has a (regular epi, mono) factorisation
  system, so by Proposition~\ref{prop:orthog-reflective-subcat}
   the full
  subcategory described in the statement of the lemma is a regular
  epi-reflective subcategory thereof.
  %Consider the inclusion \(\cat{Coclo(\mathcal{M})} \rightarrow
  %\cat{A}_{f}\).  It is a full subcategory by definition. 
  Now, the (regular epi, mono) factorisation gives the factorisation of a morphism \(h \colon A \rightarrow
  \mathcal{M}^{k}\) through its image.

  By Lemma~\ref{lem:subcat-unit-mono-closure}, the unit of the left
  hom-adjunction~\eqref{eq:hom-adj-cogen-gen} is mono at the object \(h_{!}(A)\), and
  every subalgebra of a finite power of \(\mathcal{M}\) is isomorphic
  to \(h_{!}(A)\) for some \(h \colon A \rightarrow
  \mathcal{M}^{k}\). We may conclude that 
  %and
  %it is an object in
  %\(\cat{Coclo(\mathcal{M})}\).
  \(\cat{Coclo(\mathcal{M})}\) is equivalent to a regular
  epi-reflective subcategory of \(\cat{A}_{f}\).
\end{proof}

As \(\cat{A}_{f}\)
is finitely \emph{cocomplete}, referring to Subsection~\ref{sec:reflective-subcats},
Items~\ref{it:reflection-lims}--\ref{it:reflection-co-completeness},
we also obtain: 
\begin{Cor}
  \label{cor:S-fin-colims}
  The category \(\cat{Coclo(\mathcal{M})}\) has finite
  colimits.
\end{Cor}

In light of the results leading up to here, we will reformulate our
setting in a way that suits our purposes. First, we need one last
component:

\begin{Prop}
  \label{prop:M-cogen}
  The object \(\mathcal{M}\) is a cogenerator in \(\cat{Coclo}(\mathcal{M})\).
\end{Prop}

\begin{proof}
  Consider a parallel pair of morphisms in
  \(\cat{Coclo}(\mathcal{M})\); their codomain is a subobject of
  \(\mathcal{M}^{k}\) for some \(k\in \mathbb{N}
  \). There is a projection which separates them.
\end{proof}

Consider the free and forgetful adjunction and the adjunction given by
the inclusion \(\iota\) and
reflector \(\rho\):
\begin{equation}
  \label{eq:29}
  \xymatrixcolsep{5pc}
  \xymatrixrowsep{3.5pc}
  \cd{
    {\cat{Coclo}(\mathcal{M})}
    \ar@<-4.5pt>[r]_-{\iota} \ar@{<-}@<4.5pt>[r]^-{\rho} \ar@{}[r]|-{\bot} &
    {\cat{A}_{f}}
    \ar@<-4.5pt>[r]_-{\cat{A}_{f}(\mathcal{G}, \thg)} \ar@{<-}@<4.5pt>[r]^-{\mathcal{G}\ast (\thg)} \ar@{}[r]|-{\bot} &
    {\Set_{f}}
    \rlap{ .}
  }
\end{equation}

From the results of this section thus far, we have the following.
\begin{Lemma}
  \label{lem:coclo-M-theory}
  The category \(\cat{Coclo(\mathcal{M})}\) is small, has finite colimits, and contains an object \(\rho\mathcal{G}\)
  such that:
  \begin{itemize}
  \item
    \(\rho\mathcal{G}\) is a regularly generating object;
    
  \item 
    \(\rho\mathcal{G}\) is regular projective (and so every copower
    \(\rho(\mathcal{G})\ast n \) is
    regular projective);

  \item
    Every object is a regular quotient of \(\rho(\mathcal{G})\ast n\) for some
    \(n \in \mathbb{N}\).
  \end{itemize}
  Moreover, it contains an object \(\mathcal{M}\) such that
  \begin{itemize}
  \item  Every object is a subobject of \(\mathcal{M}^{k}\) for some \(k \in
    \mathbb{N}\);
  \item
    \(\mathcal{M}\) is a cogenerator.
  \end{itemize}
\end{Lemma}


\begin{Cor}
  \label{cor:cocloMop-qv-theory}
  The category \(\cat{Coclo}(\mathcal{M})^{\mathrm{op}}\simeq \cat{S}^{\mathrm{op}}\) is
  equivalent to the theory of a quasivariety.
\end{Cor}

\section{Duality}
\label{sec:duality}

\subsection{Abstract coclones and their duals}
\label{sec:ARClo-QClo}

The results of Section~\ref{sec:quasivariety} invite the following
definitions.

\begin{Defn}[Abstract coclone]
  \label{def:arc}
  An \emph{abstract coclone} is an essentially small category
  \(\cat{K}\) which has
  finite limits and finite colimits, and contains objects
  \(\mathcal{M}\) and \(\mathcal{P}\) such that: 
  \begin{enumerate}[label=(\alph*), ref=(\alph*)]
  \item
    \label{it:arc-cogen}
    \(\mathcal{M}\) is a cogenerator;

    \item
    \label{it:arc-sub-J}
    Every object is a subobject of \(\mathcal{M}^{k}\) for some \(k \in
    \mathbb{N}\);
    
  \item
    \label{it:arc-reg-gen}
    \(\mathcal{P}\) is a regular generator;
    
  \item 
    \label{it:arc-reg-proj}
    \(\mathcal{P}\) is regularly projective (and so every copower \(\mathcal{P}\ast n\) is
    regularly projective);

  \item
    \label{it:arc-reg-quot}
    Every object is a regular quotient of \(\mathcal{P}\ast n\) for some
    \(n \in \mathbb{N}\).  
  \end{enumerate}
\end{Defn}

We resisted the temptation to call the following theories cococlones.
\begin{Defn}
  \label{def:5}
  A \emph{quasiclone} is simply a category whose dual is an abstract
  coclone. That is to say, it 
  satisfies Items~\ref{it:qv-reg-inj}
  and~\ref{it:qv-reg-subobj} in Proposition~\ref{prop:qv-flt}; has
  finite colimits; and contains an object \(\mathsf{J}\) which is a
  generator and such that
  every object admits an epimorphism from a finite copower
  \(\mathsf{J}\ast n\) for some \(n\in \mathbb{N}
  \).
\end{Defn}



\begin{Ex}[Locally finite abstract coclones]
  \label{ex:loc-fin-arc}
  Locally finite abstract relational clones are exactly those which
  arise as concrete on finite sets. 
\end{Ex}

\begin{Ex}%[Real vector spaces]
  \label{ex:vector-space}
  An abstract coclone which does not arise as 
  concrete on a finite set is given by the category
  of finite-dimensional vector spaces over the reals. As this category
  is self-dual, the category of models of the coclone and its
  dual quasivarietal theory are the same: real vector spaces. 
\end{Ex}


\begin{Defn}
  \label{def:rad-cat}
  The category \(\cat{RAdj}\) has small categories as objects and right
  adjoints as morphisms.
\end{Defn}

\begin{Defn}
  \label{def:func-o}
  There is a functor \((\thg)^{\circ} \colon \cat{RAdj} ^{\mathrm{op}} \rightarrow \cat{RAdj}\) whose
  action on objects is simply given by sending a small category to its
  opposite (dual) category and a right adjoint to its left adjoint.
  %(which is now a right adjoint in the dual).
\end{Defn}

\begin{Lemma}
  \label{lem:12}
  The functor \((\thg)^{\circ}\) gives a dual equivalence of \(\cat{RAdj}\) with itself.
\end{Lemma}

\begin{Defn}
  \label{def:subcats-radj}
  We write \(\cat{Coclo}\) for the full subcategory of \(\cat{RAdj}\)
  whose objects are abstract coclones. Similarly, we write
  \(\cat{QClo}\) for the full subcategory of \(\cat{RAdj}\)
  whose objects are quasiclones.
\end{Defn}

\begin{Prop}
  \label{prop:2}
  The dual equivalence of \(\cat{RAdj}\) with itself restricts to a dual
  equivalence \(\cat{Coclo}^{\mathrm{op}} \rightarrow
  \cat{QClo}\).
\end{Prop}

\begin{Prop}
  \label{prop:4}
  The dual equivalence between \(\cat{Coclo}\) and \(\cat{QClo}\) restricts to a dual
  equivalence between their locally finite members, that is, between \(\cat{Coclo_{Loc}}^{\mathrm{op}} \rightarrow \cat{QClo_{Loc}}\).
\end{Prop}

\section{Concluding remarks}
\label{sec:concluding-remarks}

The duality presented in this work is a \emph{syntactic} duality,
which will never extend to the entire categories of semantics as the opposite of
a locally finitely presentable category is never itself locally
finitely presentable (unless it is a partially ordered
set)~\cite{Gabriel:1971aa}. However, there are many examples of dualities where the objects with finite underlying
sets are the finitely presentable models of a coclone
theory equipped with the discrete topology (given at length in \cite{Clark:1998aa}). In these examples, the
dualising objects are a finite algebra and a finite relational
structure with the discrete topology. These correspond to our objects
\(\mathsf{M}\) and \(\mathsf{W}\), where \(\mathsf{W}\) is equipped
with the discrete topology.
The insights
gained from work for this paper will be applied to concrete dualities in forthcoming work.

An interesting line of inquiry following the work of Section~\ref{sec:sig-setting}
would be to establish whether the category of
\emph{all} models of \(\cat{Coclo}(\mathsf{M})\) is equivalent to a reflective
subcategory of \(\cat{\Psi St}\) --- not just those with finite
underlying sets. An obstacle was presented by the fact
that having a finite underlying set is not sufficient to be finitely
presentable in \(\cat{\Psi St}\). Consequently, we leave open whether
\(\FL(\cat{Coclo}(\mathsf{M},\Set)\) is equivalent to a full
reflective subcategory of \(\cat{\Psi St}\).

We stated at the outset that we would only be considering relational
clones 
%\(\Pol\)-\(\Inv\) Galois connection
on finite sets.
It should now be clear that this ensured that the finitely presentable
objects in our categories were \emph{precisely the objects with finite
  underlying sets}. However, our duality in Section~\ref{sec:duality} did not require the imposition
of any finiteness conditions. The reader may recall we began this paper by saying we provided an answer
to an invitation. We end the
paper with our own invitation: can
\(\Pol\)-\(\Inv\) duality for non-finite
sets be reconciled with ours?


\section*{Acknowledgements}
Many thanks to Richard Garner for his contributions to the proofs in Section~\ref{sec:via-sig-structures}, to Soichiro Fujii, Brian Davey,
and Ross Street for discussions which greatly benefited this work, and
to John Power for drawing my attention to the fact that this was an open problem.



\bibliographystyle{acm} \bibliography{syntax}
\end{document}
